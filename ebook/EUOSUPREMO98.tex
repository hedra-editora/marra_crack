\chapterspecial{Eu, o supremo}{}{}
 

Até oito anos eu era quieto, ia bem na escola e não fazia estrepolia, só
umas brincadeiras com meu irmãozinho mais novo.

Quando eu tinha dez anos queimei meu irmãozinho com pontas de cigarro e
não me arrependi.

Desde pequeno nunca tive amigo, mas descobri que meu irmãozinho podia
ser meu único amigo. Ele era um menino corajoso e eu era um menino
medroso.

Minha vida mudou quando meu irmãozinho caiu da laje. Foi uma tragédia
familiar. Me fizeram responsável, mas nada tenho a ver com isso.

Fiz o que devia no dia em que ele caiu da laje. E~depois da morte do meu
irmãozinho comecei a me sentir capaz de manipular as pessoas, de me ver
superior aos outros.

Naquela época descobri que gostava da minha mãe e da minha irmã. Eu me
masturbava pensando nelas, cheirava o lençol onde elas dormiam e ficava
excitado até o gozo.

Quando cresci me interessei um pouco por este tal de Freud, porque tinha
ouvido alguma coisa a respeito de filho que gosta de mãe. Achava meu
caso interessante.

Eu sempre fui agitado e nunca estava satisfeito com nada. Comecei a
fumar maconha com quatorze anos, segui na onda da cocaína, depois na
onda do crack, e me empolguei.

\asterisc{}

Sempre usei droga sozinho no meu canto. O~uso foi aumentando e não
demorou para que eu começasse a ouvir vozes e a me sentir às vezes
possuído. Tudo isso me atrapalhava ser do jeito que sempre fui. Então
busquei tratamento.

Sou sincero em dizer: tenho raiva da humanidade, e usar droga, ainda
mais a pedra maldita, me incomoda porque me aproxima da escória da
humanidade. Noventa e nove por cento dos drogados, e principalmente os
``noias'', são escória e eu, é claro, não sou e nunca fui.

Sou um manipulador de pessoas, adoro fazer isso, ainda mais com pessoas
inteligentes.

Não acredito em Deus. Admiro Hitler, Calígula, Nero e outros grandes
líderes, e faço questão de absolver todos eles porque foram grandes
homens.

Acho que existem três tipos de gente no mundo: eu, minha mulher e o
resto da humanidade. Quanto a minha mulher, eu a considero uma deusa, um
ser superior a quem venero. Em seguida vem eu próprio que não me misturo
com a ralé.

Gosto de fazer sexo com variações. Uma vez comi um cara no meio do
\emph{swing}, mesmo que transar com homem não me dê prazer e eu tenha
fantasias de ficar de quatro e dar pra outro homem.

Eu posso fazer de tudo conforme meu desejo manda. Mas as outras
fantasias de sexo que eu tenho são com mulher, e gosto de sair por aí
pra fazer sexo com putas e com garotas de programa em geral.

Aprecio sadomasô e companhia. Acho que uma mulher não tem que gemer de
gozo quando eu estou em cima dela. É~melhor ela ficar quieta porque o
que importa é o meu prazer e não o dela.

Eu não traio minha deusa só porque faço sexo pelas ruas com as putas, e
eu não me apaixono a não ser pela minha deusa. Se eu me apaixonar por
outra estarei traindo.

Considero normal sexo fora do casamento, mas se minha mulher se
apaixonar por outro homem pode ser que eu mate minha mulher, e daí? Mas
se ela transar sem paixão com outro homem pode ser que eu aceite, e daí?

Não gosto que outro homem me olhe. Não gosto que minha mulher tenha
amigo homem.

Uma vez espantei uns amigos homens da minha mulher criando terror. Eu me
fingi possuído por um demônio e disse que fazia parte de uma seita
satânica. Dei gargalhadas porque acreditaram e ficaram com medo. Adoro
controlar pessoas.

Eu me sinto Deus, já que eu não acredito em Deus como a maioria das
pessoas acredita.

Também confesso um desejo íntimo de ser pastor, porque ser pastor é uma
oportunidade de manipular. Por isso admiro um pastor que vende sua
palavra por muito dinheiro.

Tenho direito de cometer algum crime, e não porque eu seja bandido, não
faço parte dessa laia. Se eu cometer um crime, esse crime vai ter uma
justificativa porque estarei no meu direito mesmo que a sociedade
condene.

E daí que a sociedade me condene? Eu não me condeno porque me dou ao
direito de fazer o que minha vontade determina. Isso é o que importa.

Costumo sentir um tédio imenso do mundo e fico imaginando que tem muita
gente que merece ser eliminada. Por isso me sinto também justiceiro e
aprecio histórias de \emph{serial killers.}~Admiro muitos deles.

Gosto também de histórias policiais. Uma das minhas fantasias preferidas
é invadir uma casa onde tem um casal, amarrar o homem, estuprar a
mulher, e fazer um jogo com os dois, manipulando como se eu estivesse
encenando um espetáculo teatral, ameaçando cada um deles em separado,
criando terror.

Essas fantasias me visitam muito, tenho prazer com elas chegando ao
gozo, mas ainda não realizei isso, embora pense no assunto quando me
lembro excitado de quando me masturbava pensando na minha mãe e na minha irmã.

Ah, e quanto a matar alguém, eu já tentei sim, era uma pessoa
insignificante que me prejudicava no trabalho. Era um negro desprezível
que não sabia fazer nada direito. Eu tentei matar esse negro, mas não
consegui e fui mandado embora do emprego por justa causa.

Estou sempre planejando cometer um crime especial. É~o crime perfeito
que ninguém vai descobrir. E~eu jamais serei preso porque não sou
criminoso e não me misturo com gente inferior.

Eu sou absoluto, eu sou superior, chego até a pensar que não pertenço a
este mundo e que sempre busco alguma coisa melhor, sendo eu, no fundo,
também um revoltado.

Acredito que o suicídio seja uma forma nobre de conduta de um homem
superior feito eu.

Se por um mísero acaso eu tiver uma estranha percepção de não me ver
como sendo Deus ou de não me ver como sendo absoluto, pode me restar um
desespero de não me querer neste mundo e de, através da morte, me
ausentar deste mundo lixo cheio de pessoas que eu odeio.

Mas ainda bem que a verdade pode ser outra que não essa, porque quem
decide a verdade, enfim, sou eu, e porque quem quer, afinal de contas,
sou eu. Eu, o supremo.

\begin{center}\asterisc{}\end{center}
\begingroup\small

\emph{Eu considero este texto um depoimento estilizado que não chegaria
a ser uma crônica e tampouco um conto. Mas para mim o que importa é que
esse texto mostra, ao menos no âmbito do discurso, o que seria um
transtorno de personalidade antissocial.}

\emph{Trata-se, neste caso, de um diagnóstico razoavelmente seguro,
embora não se possa fechá-lo, apenas fazer uma boa suposição a partir de
um depoimento de impacto que é uma confissão cínica e chocante.}

\emph{Psicopatia não é assunto simples, embora psicopatia esteja na
moda, seja indevidamente simplificada e comumente cercada de equívocos.
Um desses equívocos é que os psicopatas seriam sempre brilhantes.}

\emph{Não é bem o caso. O ``brilho'' suspeito dos tais ``psicopatas'' é
apenas de uma minoria. E~não é a ação explícita de um psicopata que
caracteriza o transtorno de personalidade e sim a estrutura da
personalidade.}

\emph{É interessante observar que, entre a teoria e a prática da
psicopatia, acontece o inverso do que na justiça penal em que só existe
réu a partir do crime explícito e provado e não a partir das intenções
ou suposições.}

\emph{No caso da psicopatia antissocial, as intenções são evidenciadas na
personalidade antes de uma eventual ação. Existem psicopatas
antissociais ``latentes'' sem quaisquer antecedentes criminais, que
chafurdam no anonimato atrás de uma chance de barbarizar.}

\emph{Há muitos psicopatas ``adormecidos'' que partem para a ação quando
o meio e as circunstâncias são propícias, o que parece ser o caso do
protagonista deste texto.}

\emph{De qualquer maneira, eu reconheço que, neste texto, os dados de
base foram inspirados a partir de entrevistas com mais de um indivíduo
tido como antissocial, e particularmente com um deles, que conheci bem e
era bastante habilidoso com as palavras, além de ser adicto em cocaína e
crack.}

\emph{Eu me referia a ele como sendo o ``psicopata romântico'' porque eu
colocava dúvidas a respeito da realidade por detrás do seu discurso
pomposo quando ele anunciava suas intenções de perpetrar horrores.}

\emph{Fiquei meio desconfiado de seu pragmatismo declaradamente
horripilante, julgando esse pragmatismo, por vezes, uma ficção
anunciada. Ainda mais porque, afinal de contas, o que se impõe em certas
psicopatias é menos uma prática de ações explícitas e mais uma verdade
psíquica. Essa verdade pode vir no curso da fantasia ou mesmo da
fabulação. Mas, apesar dos pesares, no caso dele não restam dúvidas de
que ele é um tipo bastante perverso.}

\emph{Não creio, todavia, que ele estivesse fazendo apenas ``teatro'',
embora seja possível que eu tivesse me deixado impressionar por algumas
possibilidades imaginativas e teatrais de perversidades ainda não
testadas, talvez, na prática. Vou até um pouco mais longe. Creio que
ele, a bem da verdade, flutuasse entre suas imaginações e suas intenções
perversas.}

\emph{Eu me lembro de que ele veio a tratamento não para se livrar de
seu perfil psicopático antissocial. Ele percebeu que sua adicção estava
incomodando demais, porque ele ouvia vozes e sofria de episódios de
psicose cocaínica.}

\emph{Então ele queria se livrar dessas perturbações para ficar com sua
personalidade mais ``limpa'' e estar mais livre para exercer suas
monstruosidades.}

\emph{Nota-se ainda que existe no discurso dele um flerte descarado com
a figura do grande herói, embora seja a figura do grande herói invertida
e satanizada descambando no que seria idealizado por ele como a figura
elevada de um homem ``superior'', e sendo nada superior porque, lá no
fundo, a verdade é outra.}

\emph{Seu ``homem superior'' é uma imagem distorcida dele --- um pobre
diabo terrivelmente obcecado em fazer de sua vontade a única lei dentro
de um egoísmo e narcisismo supremos.}

\emph{Encontram-se exemplos de tudo isso na História. Acrescento que não
faltariam aí comparações com a personalidade de certos monstros
nazistas, na medida em que o nazismo é uma satanização romântica.}

\emph{Não faltariam também referências psicanalíticas de fantasias
perversas que se entregam a imperativos sádicos do desejo brutal sem o
freio de qualquer superego. E~digo mais ainda: todos esses conteúdos
psíquicos dele podem ter alguma sintonia com a sombra de todos nós,
sendo justamente por isso que a figura do psicopata é muito cênica e
fascina.}

\emph{Não é por acaso que existem tantos filmes de sucesso sobre
psicopatas, e quase sempre os antissociais! Basta verificar nas
locadoras.}

\emph{Talvez, lá no fundo, um psicopata seja aquele que se dispõe a
realizar nossas escondidas monstruosidades.}

\emph{Nelson Rodrigues tem uma frase linda que leva a uma reflexão
interessante: ``o sujeito, seja ele um homem de bem ou um pulha, é um
assassino falhado. Não há ninguém, vivo ou morto, que não tenha
concebido a sua fantasia homicida''.}

\emph{Portanto, o cinismo e a arrogância deste rapaz feito personagem
dele mesmo e beirando alguma insanidade passageira ecoa junto à falsa
liberdade de ele querer ser ``supremo'', de querer ser um deus
debochando de Deus e, ao ser ateu, negar e destruir para construir na
negação.}

\emph{Eu acredito que esse depoimento estilizado pelo horrorizante
cinismo chegue próximo do que se pode ter como literatura e também, quem
sabe, como provocação ou ponto de partida para reflexões filosóficas.}

\emph{Para finalizar exponho a seguinte linha de argumentação: nas
ciências psi existe uma polêmica se o transtorno de personalidade
antissocial seria ou não uma ``doença''.}

\emph{Há quem diga que essas pessoas não sejam doentes, apenas têm
condutas variantes socialmente perniciosas. Elas não seriam doentes
porque nelas não haveria disfunção. Pelo contrário, podem estar
``saudáveis'', embora voltadas para o que consideramos ou julgamos como
sendo o mal.}

\emph{Essa postura, no entanto, é discutível. Essas pessoas --- como eu
conheço muitas --- chafurdam no lodo do desejo irrealizado e podem
sentir o peso da solidão e da amargura. Não creio que, em geral, sejam
felizes e, portanto, questiono se essas pessoas não seriam
disfuncionais. Ou talvez algumas sejam disfuncionais e outras não.
Porque há psicopatas que realizam seus desejos. Outros que não.}

\emph{Ou sei lá eu se haveria ou não doença. Confesso humildemente que o
assunto é bastante complexo e que não tenho conclusões definitivas. Mas,
de qualquer maneira, este assunto é ao mesmo tempo fascinante,
profundamente perturbador e transcende ao que seria meramente
científico.}

\emph{Este assunto desemboca na seara complexa dos valores que a vida
atribui a si mesma. Eis porque eu busco a literatura, e sei que a
literatura, como o teatro, vive de conflito e de tensão. É~onde sempre
vem à tona o velho jogo do bem e do mal, de modo a mexer com nossas
esperanças de soluções para com a vida.}

\emph{Eu ainda diria que todos nós temos um gostinho especial de
acompanhar o herói às avessas, o herói mau que nega e detona tudo. Não é
à toa que, no teatro, o público se lembra mais dos personagens maus do
que dos bons.}

\emph{Pense sobre isso, leitor, antes que eu sugira a você a seguinte
provocação: não é bom ter a companhia de um psicopata, mas dá prazer
vê-lo numa tela enquanto se come pipoca, ou é confortável tomar contato
com ele lendo um texto bem escrito.}

\emph{Tudo isso pode ser edificante, porém é só edificante na manha e em
termos, é claro. Em resumo: conhece-te a ti mesmo até nos meandros mais
sombrios alheios; os outros,~que para o Sartre são o inferno, também
servem um pouco de espelho, por mais distorcido que seja esse espelho.~}
\endgroup