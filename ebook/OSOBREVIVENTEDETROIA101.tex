\chapterspecial{O sobrevivente de Troia}{}{}
 

Era um rapaz estranho, diferente dos outros, e não apenas por ser cego
de um olho. Ele tinha algo mais.

Sua primeira aparição foi numa aula de teatro para dependentes de
drogas.

Ele parecia ser o mais perturbado e o mais problemático.

Eu tinha preparado uma introdução sobre as origens do teatro, e comecei
perguntando a todos o que seria uma tragédia. As respostas foram banais.

Introduzi o termo sacrifício, e perguntei a todos o que era. As
respostas foram banais.

Parti do fio guia do que é uma tragédia comum e fiz depois uma ligação
sutil entre essa tragédia comum e a saga do herói. A~seguir situei o
sacrifício como um resgate que tem origem em uma ultrapassagem e em uma
ousadia, justificando a intervenção do Destino e dando pleno sentido ao
ato do herói.

E resolvi contar a história de Édipo porque é uma história instigante, o
enredo é rico, sedutor, universal e é garantia de sucesso de público.

Todos prestaram muita atenção, ainda mais porque não conheciam a
história, nem mesmo através da repercussão de uma novela televisiva
passada há vários anos.

No momento mais tenso do enredo cheguei ao enigma da esfinge e à
pergunta-chave:~``qual é o animal que de manhã anda de quatro, ao meio
dia anda de dois, e de tarde anda de três?''

Houve um intervalo de silêncio. Eu arrisquei dizer que ninguém
adivinharia, pois a história seria, para eles, uma novidade.

Refiz a todos a pergunta clássica da esfinge a Édipo.

Novo silêncio.

Quando eu ia quase revelando o enigma, o rapaz cego de um olho me deu a
resposta fazendo-me também uma pergunta:

— Não é o homem?

Olhei para o rapaz duplamente no seu olho cego e no seu olho vivo e dei
a ele os parabéns, e disse a todos que, se eu fizesse ali o papel da
esfinge, teria, de alguma forma, que sinalizar que me jogaria no abismo.

\asterisc{}

Aquele rapaz depois começou a participar do meu grupo de teatro, mas
sumiu por um tempo. Num segundo momento, quando eu menos esperava,
reapareceu.

Na ocasião eu havia proposto ao grupo um exercício: era uma entrevista
imaginária a Dionísio, em que o próprio Dionísio conta sua
``biografia''. A~entrevista era lida por todo o grupo, disposto numa
roda, sendo que cada um era um pequeno ``Dionísio'' e eu o
entrevistador.

Convidei o rapaz de um olho cego para participar, mas ele se perdia na
leitura. Tinha dificuldades. Não conseguia acompanhar os outros. Achei
que ele deveria ficar fora do exercício ou que talvez não compreendesse
nada. Mas ele ouviu muito bem a leitura dos outros.

No final do exercício, ele tomou para si a história mitológica e, de
repente, fez dela uma adaptação improvisada. Recriando tudo, ele
metamorfoseou Dionísio e os personagens que completam seu entorno
familiar e até olímpico.

Todos adquiriram nomes brasileiros, e nomes da Zona Leste de São Paulo,
junto às ``quebradas'', às favelas e seus entornos.

Ele adaptou o menino Dionísio do mito, que virou um brasileirinho comum
nascido na favela, atolado nas vicissitudes sociais problemáticas da \versal{ZL} de Sampa, 
e teve um mentor alcoolista frequentador de boteco.

Esse menino, mais tarde, começou a usar droga. Traficou. Saiu por aí a
reclamar a mãe morta em tragédia doméstica. Reclamou também do paizão
poderoso indiferente e ausente. Finalmente, desceu aos infernos para
resgatar sua mãe e sua história de vida.

Eu dei a ele os parabéns pelo trabalho original. Trabalho, aliás,
surpreendente!

\asterisc{}

Na terceira vez em que ele apareceu foi tudo um franco desatino. Ele
chegou tomado daquilo que os psiquiatras podem diagnosticar de mão cheia
como mania.

Não havia limites ao seu entusiasmo ensandecido. Estava aceleradíssimo.
Achava que conseguia voar. Disse ser do \versal{PCC}, o que claramente não era verdade
Ainda fiquei sabendo que ele havia assaltado uma biqueira, e
que de lá retirara inúmeros pinos de farinha branca e de crack, além de
maconha e distribuiu tudo e também usou de tudo.

Os traficantes fizeram um rebuliço e já iam jurá-lo de morte. Mas a
ousadia foi tanta que os traficantes relevaram o ato. Como se dá muito
neste país Brasil, a encrenca foi resolvida no jeitinho.

Ele obteve um ``deixa pra lá'' do mundo das sombras, ainda mais porque
alguém repôs as perdas. A~ofensa foi apagada pela sandice quixotesca de
um ``noia'', e ele foi salvo pela sua obsessão e pelo velho acordo que
costuma selar uma trégua na fúria dos homens.

Eu, cautelosamente, o adverti dos perigos.

\asterisc{}

Depois foi a quarta e última aparição.

Na época eu estava fazendo, com uma atriz amiga, uma apresentação de uma
peça minha, que é uma adaptação da Odisséia em narrativa dramática para
dois atores, dentro daquilo que eu tomaria a liberdade de chamar de uma
peça clássica em ``linguagem de mano''.

Para fazer uma demonstração desse trabalho, naquele dia escolhemos uma
parte pequena da Odisseia, e que fala do início da guerra de Troia.

Fizemos uma apresentação meio improvisada usando máscaras e adereços
comuns, como bonés e óculos escuros \emph{rayban}, para recriar um ``clima'' de
mito e aventura.

Havia uma pequena plateia bem interessada. O~rapaz com um olho cego
prestava bastante atenção e, no final, tomou-se de grande entusiasmo e
fez questão de ter um diálogo comigo.

Ele não estava possuído por nenhuma mania. Estava bem lúcido e calmo.
Mas estava visivelmente emocionado e fez questão de abrir sua
comunicação com uma frase muito forte, impactante, confessional:

— Minha vida é a Guerra de Troia.

Eu disse que compreendia bem que a vida poderia ser uma guerra. No
entanto, ele respondeu que sua história era uma guerra particular.

Ele então foi dando nomes aos bois.

Há alguns anos ele e outros companheiros faziam parte de um mesmo bando.
O~irmão dele, Cristiano, era o líder. Mas havia outro bando em outra
favela, ou outra ``quebrada'', que tinha outro líder.

Cristiano era o Menelau. Havia uma tal de Janete, e Janete era a Helena
de Troia. Havia o grande rival de Cristiano, líder do outro bando e da
outra ``quebrada''. Era o Páris, cujo nome real ele não citou.

Eu ouvia uma história absolutamente realista, terra a terra,
brasileiríssima e da \versal{ZL} de Sampa, como muitas outras histórias
semelhantes em que bandos disputam territórios e presas. Nada mais
universal, puxa vida!

É claro que a ``guerra'' teve um estopim, e foi quando Janete/Helena foi
raptada por este Paris, e Cristiano/Menelau, com seu bando, seguiu atrás
dos raptores.

Tal como seria esperado, Cristiano/Menelau conseguiu, mediante vários
ardis, infiltrar-se na ``quebrada'' inimiga. E~uma operação final de
cerco lembraria uma operação Cavalo de Troia.

Cristiano enfrentou cara a cara seu inimigo. Ali na raça, mano, tá
ligado! O rapaz chegou a admitir que a cena teria sido quase do tipo
``faroeste caboclo''.

Cristiano levou a melhor. Tanto assim que um casaco de couro do inimigo
ficou varado de balas. E~quando Cristiano voltou para sua ``quebrada''
carregando Janete, ele proclamava a Deus e ao mundo:

— Matei o tigre e arranquei o couro.

Mas a história não morreu ali. A~guerra continuou.

Todos, de ambos os bandos, foram tombando, um a um, na valeta comum e
recolhidos ao vale das sombras. A~começar quando houve uma
``crocodilagem'' ou, dizendo de outra forma, quando houve uma
``trairagem''.

Ao fazer essa revelação, o rapaz não sorriu de malícia. Abaixou a cabeça
e confessou que pegaram Cristiano pelas costas e o deixaram varado de
balas.

A tragédia largou para trás os dias heroicos e assumiu o sentido da
miséria comum e vulgar. O~resto foi inglório.

De todos os que participaram da guerra, sobreviveu ele, com a marca de um
tiro que lhe vazara um olho. Daí ele se viu sozinho no mundão sem
pertencer a nada e a ninguém. Começou a fumar a pedra de crack, e ficou
perambulando pelos becos e vielas da Leste e de Sampa.

Eu fiz um minuto de silêncio.

\asterisc{}

Depois daquela aparição ele foi embora de vez, e eu então me lembrava
novamente do enigma e da pergunta da esfinge.

E se, diante da resposta correta dele, desejei ser de novo a esfinge
para me atirar no abismo, ao cair no abismo desejei cair nas mãos de
Deus, tal como está escrito numa das mais belas frases de Nietzsche.

\begin{center}\asterisc{}\end{center}
\begingroup\small


\emph{Em outro texto deste livro eu escrevi que o cronista precisa ser
fiel (em termos) à realidade, porém o ficcionista pode ser um
``enganador'' articulado por falsear artisticamente a realidade.}

\emph{Então vou confessar uma verdade de escritor: esta história é um
conto no qual foram acrescentados dois, ou melhor, três pontos.}

\emph{O~rapaz de um olho só existe. O~que tinha muitas perturbações e
respondeu brilhantemente à pergunta da esfinge também existe. O~que teve
uma crise maníaca e assaltou uma biqueira também existe. São três
protagonistas, porém reunidos em um só.}

\emph{Todos são adictos graves, todos ``buscadores de novidade'',
dionisíacos e propensos ao teatro, e todos suficientemente ensandecidos
no mundo da adicção pesada de tal maneira que suas aventuras
mirabolantes e quase absurdas não constituem grande novidade para quem
trabalha com drogados na periferia.}

\emph{Mas quem escreve a respeito deste mundo sabe que, para se fazer um
conto a partir de uma crônica é preciso buscar um elemento
representativo, um indício especial, ou um detalhe particular que
represente o conjunto.}

\emph{Pode ser o caso desta fórmula de ``três em um'' que vem a calhar.
Porque eu juro que os três protagonistas se parecem como se fossem
irmãos de ``ativa''. Eles têm em comum classe social, infortúnios
familiares, abandono, miséria e o diabo.}

\emph{Por isso mesmo eu digo que esta síntese de protagonistas facilita
conclusões também para levantamentos diagnósticos, no sentido de se
encontrar pontes comuns entre a miséria social, o viés individual e a
busca infinita da ``brisa''.}

\emph{Acontece que este assunto é um domínio humano movediço dentro do
qual quase sempre não é possível encontrar doenças, porém sofrimentos
difusos e algumas síndromes.}

\emph{Estas últimas podem ser escancaradas, tal como é o caso da
síndrome maníaca e também, conforme uma liberdade de expressão, de uma
``síndrome da disfunção familiar ou social''.}

\emph{A~dar crédito ao que o rapaz de um olho só me contou, sua vida tem
semelhanças com todas as guerras deste mundo, inclusive com a guerra de
Troia, apesar da antiguidade desta guerra, e apesar de sua mitificação
estar muito além da realidade histórica.}

\emph{Como se pode facilmente perceber, esta outra ``guerra'' do rapaz e
seu bando --- uma guerra menor tida como torpe ou miserável --- seria
muito provavelmente um confronto entre grupos ligados ao narcotráfico
disputando não apenas território, mas também grana e belas e sedutoras
``gatinhas''.}

\emph{Mas é aí que a semelhança com o passado remoto fica para mim
contundente e dramática. É~aí que as glórias humanas perdem para a
imperiosidade do desejo comum, e eu digo cinicamente para os meus
botões: puxa vida, como a humanidade é a mesma em milhares de anos, e o
que é Troia no imaginário de todos nós, ou o que são as várias
``troias''?}

\emph{Acabo admitindo que os ``clássicos'' da literatura apenas reprisam
a mesma história. E~do quanto melhor reprisam, mais essa história se
aproxima de um modelo universal. Como em uma frase do Tchekhov: ele teria
dito, se não me engano, que ao descrever bem sua aldeia você está
descrevendo o universo.}

\emph{Pois é! Eu acredito que toda boa arte do conto faça esta ponte da
aldeia com o universo. E~reconheço agora uma verdade singular: o que me
levou realmente a escrever o conto-crônica foi o exato momento em que
fiquei impactado com a frase do rapaz: ``minha vida é a guerra de
Troia."}

\emph{De uma certa forma, é claro que não posso ter certeza das supostas
verdades factuais do que ele me contou a seguir, mas creio que ele tenha
sido bem realista porque não tinha mais a perder, porque precisava de
uma escuta, porque queria desabafar.}

\emph{O~importante é que ele criou um paralelo notável naqueles
personagens bem brasileiros que de uma hora para outra assumiram nomes
dados por Homero.}

\emph{E ele foi performático e convincente ao dar nomes aos bois e ao
fazer aquela transposição da antiguidade para a atualidade --- nomeando
helenas, menelaus, páris e companhia limitada.}

\emph{No entanto, muita gente pode dizer dele que se trata de mais um zé
ninguém, de mais um ``noia''. De fato, ele é mais um dependente químico
anônimo circulando na metrópole paulistana.}

\emph{Há vários como ele (ou eles), e é cômodo olhar para ele (eles) sob
uma ótica médica ou psiquiátrica reducionista, como também é fácil e
cômodo olhar para ele (eles) a partir do estereótipo criado pelo
preconceito.}

\emph{Mas acontece que tudo é muito complexo neste domínio do desejo
obsessivo, e cada um, lá no fundo, tem suas razões.}

\emph{Ninguém é totalmente inocente. Ninguém é totalmente culpado.
Ninguém é igual a ninguém.}

\emph{Portanto, em meio a essa relatividade a gente se pergunta se as
perturbações dele vieram da droga, da vida louca, do crime, ou se vieram
de suas particularidades existenciais.}

\emph{Não existe resposta clara. Sobram hiatos e espaços de dúvida.}

\emph{Esta história mostra, ao menos para mim, como uma certa ``loucura''
(loucura é um termo tecnicamente duvidoso ou impreciso) pode conviver
com uma certa lucidez e criatividade. Embora seja uma falácia dizer que
a ``loucura'', no sentido da psicose franca, faça as pessoas livres,
felizes ou criativas.}

\emph{Há um sofrimento real nessas pessoas, embora elas possam de
repente surpreender com observações brilhantes. E~eu ainda diria que as
observações brilhantes são tão verdadeiras quanto existe, entre muitos
tipos tidos como ``noias'', uns ``buscadores de novidade'' notáveis ou
uns anti-heróis a nos contar sagas mirabolantes por detrás de disputas
até torpes do narcotráfico.}

\emph{Enfim, como muitas destas pessoas chafurdam no submundo, elas
expõem o desejo humano bruto às escâncaras, elas sentem na carne as
arestas da existência em mundo cão e são capazes de dizer uma frase
dessas: ``minha vida é a guerra de Troia''.}

\emph{Como outras pessoas diriam, se tivessem condições, ``minha vida é
uma Odisseia'', ou ``minha vida é uma Divina Comédia'', ou ``minha vida
é Dom Quixote de La Mancha''.}

\emph{Portanto, novamente a mitologia vem cumprir seu papel junto à arte
da palavra, e a questão precisa do diagnóstico fica um pouco em segundo
plano.}

\emph{Não é à toa que eu me pergunto: que certeza eu tenho de toda esta
história?}

\emph{Bem, digo que tenho apenas uma pálida certeza de que este rapaz
(ou os três rapazes que fundi no protagonista) é bastante perturbado,
tem rompantes maníacos, e apresentou momentos de lucidez e criatividade.
Param aí as certezas.}

\emph{E não quero dizer que ele, ou que todos eles, sejam ``bonzinhos''
ou ``coitadinhos'' e nem que devam receber o chavão de serem apenas
``vítimas da sociedade''. Tolice. Destaco apenas, no caso específico
dele, as contradições vivas e as perturbações no curso da vida como ela
é ou me parece ser e no pano de fundo complexo da uma existência lascada
nas ``quebradas'' da Zona Leste de Sampa, fazendo eco aos escritos de
Homero que nos dão lições da universalidade da condição humana.}

\emph{Então, caro leitor, eu ainda me reporto àquela fantasia que tive
no dia da aula de teatro. De repente eu mesmo, como improvisado ator e
diretor, posso ser a esfinge e desejar me atirar no abismo para cair nas
mãos de Deus, como está numa das mais belas frases de Nietzsche.}

\emph{\asterisc{}}

\emph{Segue, agora, o que era para ser o final do conto-crônica, que
resolvi retirar do texto principal e colocá-lo aqui, nesta tentativa
reflexiva de terminar minha confissão pessoal dentro deste} making of
\emph{improvisado.}

\emph{Acontece que, às vezes, a gente pode ser salvo pela literatura e
pela arte. O~resto parece ser uma agonia e uma grande incerteza.}

\emph{Fiquei com toda aquela história na cabeça matutando. Uma pergunta
semelhante à pergunta da esfinge ainda permanece como dúvida perene.}

\emph{Todo mundo sabe que, quando Troia foi arrasada, um certo Enéias
fugiu de lá. Enéias teve um destino glório porque os deuses lhe
propiciaram a tarefa de ter dado início a um longo processo que,
mitologicamente, culminou na fundação de Roma.}

\emph{Eu não resisti a me fazer uma pergunta: aquele rapaz, o das várias
aparições misteriosas, afinal de contas, estaria ele, em seu pequeno
destino, fundando ou erguendo o quê?! Qual seria o processo em curso?}

\emph{Ainda mais estando perdido aqui na \versal{ZL} de Sampa, movendo-se nas
suas ``quebradas'' que são bastante realistas, ainda mais sendo ele tão
terra a terra, tão sobrevivente, como tantos outros, de uma outra guerra
pequena quase invisível ao mundo.}

\emph{Retorno à questão inicial da busca infinita de uma certa ``brisa''
e torço para que este rapaz apareça novamente e tenha um instante de
``iluminação''.}

\emph{Mas fica suspensa no ar ou, quem sabe, fica em trânsito, uma
certeza relativa de que a última aparição dele é uma ausência a ser
preenchida.}

\emph{Uma resposta possível é que estamos sempre procurando responder a
perguntas de muitas esfinges, somos também as esfinges, e este é um
caminho, um caminho lúcido também pontuado de perturbações e de buscas
que nunca se acabam…~}
\endgroup