\chapterspecial{Antônio Santiago}{O dito e o oculto}{}
 

— Sou da Paraíba, sim senhor, e meu nome é Antônio Santiago. Meu finado
pai, Deus o tenha, foi matador. Eu vim do Norte, pequenino, pra São
Paulo que é grande demais. Depois voltei pra Paraíba e, de novo, pra São
Paulo e aqui fiquei.

Ele lança um olhar vivo, suplicante e bastante inquieto, embora tenha um
domínio amplo da situação e fale com eloquência popular, bonito e
sonoro, como se estivesse numa prédica ou comício.

— Você usa droga?

— Ora, homem, de tudo e comecei cedo. Dizem que sou um tal de adicto,
mesmo porque agora estou direto na pedra e não escondo essa verdade.

— Aprecio a sua sinceridade.

\asterisc{}

Antônio Santiago acaba de surgir caminhando pela Rua do Gasômetro como
um pedinte comum, quando ele então se entusiasmou diante da janela do
meu carro parado numa solidão só junto ao meio"-fio.

— Minha vida, seu doutor, sempre foi muito sofrimento desde quando eu
morava na Paraíba, onde meu irmão se enforcou e eu fiquei com sua morte
na consciência me achando culpado. Carrego até hoje essa cruz.

Não existe um bando uniforme de gente morando na rua, existe um e outro
ao Deus dará e cada qual diferente com seu drama particular. Sendo do
Norte do Sul do Leste ou do Oeste, com sotaque identificado ou não, o
mundo dos esquecidos pode ser surpresa, mesmo quando se espera o de
sempre, e a gente sempre espera.

— Você tem família?

— Tive casa e tive carro. Hoje a família não me quer. Mas já fiz um filho
que vive com quem foi minha mulher, lá no interior.

O homem que se diz chamar Antônio Santiago e seria filho de matador me
lembra aqueles tipos descritos por Euclides da Cunha.

E você, aí, que me lê, já leu Os Sertões? Pois leia bem, meu irmão.
Prenda"-se atentamente ao mundo mágico e guerreiro de Antônio
Conselheiro. Repare nos tipos suspeitos e fascinantes que o cercavam ---
aquela suposta marginalia de gente mal costurada nos suplícios da terra,
indo do afeto transbordante do bem à perversidade nada serena do mal.
Misturando o que se alega fanatismo religioso a tendência criminosa. Com
índole de se colocar em marcha neste mundo, ser andarilho, buscar
utopias e terras prometidas.

— Sou mendigo, sim senhor, não com orgulho. Tenho somente a roupa do
corpo porque me levaram os pertences e os documentos. Aqui estou lhe
pedindo ajuda porque é minha necessidade, e se fui obrigado a aprender a
fazer isso bem, hoje consigo tirar moeda até de polícia.

Na época do Conselheiro o Sul era distante e quase inacessível. Para
muita gente a República era amaldiçoada e nos sertões se amava a figura
lendária de um rei -- Dom Sebastião ou algum outro Messias.

— Tenho vergonha de incomodar o senhor, que me escuta com boa vontade. E~que me comove, eu juro. Mas só peço uma ajudinha. Não minto e não
engano. Não é fácil, como o senhor bem sabe, morar na rua, ser mendigo.

O que se discursa bem articulado é político e interesseiro neste mundo
de Deus e do Diabo, até durante o ato de pedir esmolas. E~ser morador de
rua é um ato de complexidade shakespeariana.

\asterisc{}

Olho em volta na solidão do início da noite. Percebo que São Paulo está
cheia dessa gente. Muitos são como Antônio Santiago, brasileiro nato e
filho de matador, de olhar fuzilando entre o bom humor e a malandragem
suspeita.

— Você pode falar à vontade que estamos aqui ouvindo, e não é bom para
você encontrar alguém que escuta?

Antônio Santiago estende os braços para baixo, olha bem para mim, e bate
as mãos sobre as bermudas gastas e sujas.

— Apenas uma ajudinha, seu doutor.

Eu sei que quase todos eles costumam ser lúcidos apesar da cachaça ou de
outra droga, que usam de combustível! Muitos são bons discursadores e
seriam, outrossim, políticos de plenário ou pastores evangélicos. Sempre
preenchendo a atenção de um casual espectador com seu palavreado
recheado de floreios. Ou são atores que nunca perdem o humor nascido no
tom da voz, no puxar do cantado no final da frase, no arrastar sonoro do
sotaque do Norte e Nordeste do Brasil. Nem dispensam elucubrações
metafísicas entre Deus e o Diabo nesta terra do sol, mesmo que o façam,
e bem feito por sinal, à maneira de literatura de cordel.

— Apenas uma ajudinha, seu doutor. Em nome de Nosso Senhor que está no
Céu e vela por nós aqui em baixo.

A religião, para eles, tem cheiro da terra -- é luta eterna entre
potências misteriosas no meio daquilo que faz os pobres humanos virarem
marionetes sob o jugo dos deuses celestes e dos deuses das sombras de
mundos inferiores.

— Humildemente peço uns trocados, que não lhe fazem falta.

Nós, supostamente de uma ``elite branca'' e burguesa, estamos apenas
confortáveis dentro de um veículo moderno que exibe riqueza; somos uma
gente estranha dando sopa na solidão perigosa da baixada próxima ao
Parque Dom Pedro no início da noite, e ainda somos aqueles que um dia
fizeram parte do mundo dos coronéis, e assim podemos ser reverenciados
ou odiados.

— Como o senhor acabou de apreciar minhas simples palavras e minha
sinceridade, não escondo em lhe dizer o que pensam de mim: apenas um
pobre diabo que chamam por aí de drogado ou de noia.

Neste momento eu me sinto como uma insegura plateia em um teatro de rua
improvisado, ou em um quase cinema.

Vejo um ``documentário'' ao vivo, e se tivesse uma câmera colheria uma
amostra de explosiva brasilidade.

— Ou então, seu doutor, eu posso ser mesmo um lixo, feito este vosso
servidor, aqui em carne e osso, que o senhor vê na sua frente.

É verdade que muitos deles estão sempre nascendo e muitos estão sempre
morrendo. E~se alguns são repetitivos e medíocres, outros são criativos
e com histórias para boi dormir ou para boi acordar.

Mas diante deste arremedo de show de rua, tiro minha carteira e, num
capricho de momento, puxo uma nota de vinte reais. Não conto, porém, com
a surpresa de Antônio Santiago quando ele tateia a nota e me olha
atônito por prolongados segundos.

O dinheiro é prêmio de consolação para a rotina de um dependente de
crack esmolando moedinhas, ou é um excesso de esmola da minha parte, ou
é um gesto semiconsciente de prevenir um assalto.

Um súbito espanto, porém, de Antônio Santiago me torna constrangido. Eu
me imagino como um burguês arrogante em gesto de abuso de poder para me
divertir. Ou então só compartilho minha alegria dando o dinheiro.

Antônio Santiago não perde o rebolado e finaliza seu discurso floreado.
Mantém a mesma lucidez drogada ou embriagada de quem quase estaria
disposto a criar um pouco de ficção em sua literatura oral. E~a tal
ponto que ele já completara seu texto e já deseja ir embora --
verdadeiro ator em despedida.

\asterisc{}

Ele vai se retirar de cena com um cachê relativamente polpudo e
surpreendente depois de ter feito sua bonita palavra nordestina
triunfar. Mas Antônio Santiago se permite um momento final de gratidão,
sabedoria e sutil malandragem, e ele o executa com um toque de
autoengano que é o lado torto da autoajuda suspeita.

Ele estende diante de mim a nota de vinte reais testando minha
disposição em tomá"-la de volta. Percebo a intenção malandra, pois ele
sabe que eu sei o que significam os vinte reais para mim e para ele. Ele
teria intuído não ser politicamente adequado da minha parte pegar a nota
de volta. Certíssimo. Eu jamais o faria, é claro, até por orgulho
próprio.

Então Antônio Santiago se dá o aval de confirmar a posse do dinheiro e
ficar livre para gastá"-lo. Sorri torto e ironiza a possibilidade da
compra de duas ou quatro pedras de crack. Depois corrige ao declarar a
intenção de adquirir um par ordinário de sandálias.

Minha consciência torce para que isso seja verdade, enquanto fico em
dúvida se eu teria sido um pouco sádico. Afinal de contas, acabo de
provocá"-lo colocando um objeto do desejo nas suas mãos. E~ele sabe que
eu sei que ele pode ir até a biqueira adquirir as pedras para satisfazer
uma fissura passageira, ou pode beber até cair, ou lamentar com os
míseros vinte reais sua pobreza extrema.

Eu olho para o relógio, sinto que o tempo corre, olho melhor para
Antônio Santiago e me pergunto, com um ranço de pessimismo e sabendo
estar num lugar perigoso: será que se encarnaria nele um pouquinho do
rancor de um jagunço do Conselheiro, que tinha ódio aos coronéis embora
os reverenciasse? Seria ele o vingador e, ao mesmo tempo, o beato em
busca de redenção?

Mas, para meu alívio, Antônio Santiago, da Paraíba, filho de matador,
simplesmente se curva todo solene para agradecer e prossegue caminho no
escuro da Rua do Gasômetro.

Logo depois aparece outro morador de rua sem retórica e brilho, um tipo
comum que vem com a conversa de sempre. Entediado, eu lhe dou uma
moedinha. O~morador de rua vai embora no sentido Parque Dom Pedro.

Esse outro não parece fazer parte do mundo esquecido de Euclides da
Cunha. Ou, quem sabe, fizesse? Porque os sertões, caro leitor,
mudaram"-se para São Paulo. E~porque uma guerra continua. E~bandidos e
santos estão por aí. Sublimes e malditos. No entrelaçamento do bem e do
mal e atrás de satisfazer necessidades normais ou de se perder em busca
de objetos do desejo como faz qualquer vil mortal. Ademais, nunca
ninguém sabe o que se passa de fato na mente dos outros.

O tempo da nossa permanência aqui se esgota e outras figuras humanas
emergem das sombras no meio desta solidão. Enfim, o melhor que fazemos,
nesta hora suspeita, é jantar num bom restaurante no Brás onde vinte
reais pagam o \emph{couvert}.

Forramos o estômago e falamos sobre Antônio Santiago, para depois eu me
lembrar dele aqui, nesta crônica. E~ainda bem que me lembro porque é
para isso que serve a literatura e é por isso tudo que eu continuo a ter
vontade de escrever.~

\begin{center}\asterisc{}\end{center}


\emph{Eu diria que um olhar atento para um morador de rua é um convite
para rever um estereótipo e, ao mesmo tempo, é uma oportunidade de
constatar o que pode ser estranho, inusitado, exótico ou algo mais.}~

\emph{Quando se trabalha com dependentes de drogas, há uma diferença
fundamental entre atender pessoas dentro de um ambiente institucional e
protegido e, por outro lado, encontrar pessoas na rua ao Deus dará ou
num joguete entre Deus e o Diabo perambulando ao sabor do acaso e da
necessidade. Principalmente quando se trata de alguém no embalo da onda
moderna do crack --- esta droga perigosa, mal compreendida e tão
demonizada.}~

\emph{Meu encontro com Antônio Santiago (nome parcialmente fictício)
teve a peculiar circunstância de haver ocorrido no início da noite e na
solidão próxima ao Parque Dom Pedro, e de haver iniciado no calor da
espontaneidade.}~

\emph{Foi quando se produziu a aparição de um morador de rua com um
jeito especial, e quando se construiu um momento criativo através da
cenicidade e da palavra sonora bem colocada. Porque ele era certamente
um desses tipos que se valem da retórica popular nordestina em sua
proposta sedutora para vender o peixe, como se diz popularmente.}~

\emph{Confesso ter ficado com receio, ainda mais devido à violência
presente na cidade e sendo o local de encontro ermo. Mas fui pego de
surpresa pelo rompante da apresentação dele, e sua teatralidade foi
sutilmente cedendo espaço a uma manipulação malandra e interesseira.}~

\emph{De repente virei cúmplice de mim mesmo quando soltei mais do que o
dinheiro habitual a um morador de rua, e seja para me defender de algum
temido assalto ou, de maneira um pouco sádica, para me divertir com o
que me pareceu um pequeno e improvisado show.}~

\emph{E aproveito agora o ensejo para comentar um fato que julgo
importante. Quem convive com drogados em tratamento costuma ignorar o
universo do uso social da droga; costuma ignorar o conjunto abrangente e
desconhecido daqueles que não procuram ajuda e mal e porcamente vivem na
contra mão da vida buscando efêmeros prazeres entre amarguras que não
costumam ser efêmeras.}~

\emph{É, portanto, nesses momentos crus, impactantes e espontâneos que
os recursos acadêmicos --- sejam médicos dentro dos centros de
atendimento à saúde, ou sejam os recursos das ciências humanas em geral
--- podem dar com os burros n'água. Isso pode ocorrer na realidade
anárquica de rua e na grande cidade brasileira, quando uma droga
qualquer (seja o crack ou a mais onipresente cachaça) é mais um elixir
banal sem grande novidade no front.}~

\emph{O~interessante é que, no caso dele, tudo foi permeado por uma arte
dramática fugaz de ocasião. Uma arte retórica digna de uma literatura de
cordel, que pode surgir anônima em qualquer canto da cidade e
desaparecer feito papel jogado ao lixo.}~

\emph{De uma maneira ou de outra, acho que a perspectiva digamos assim
mais ampla da injustiça social e do crime apareceram sutilmente no show
de rua de Antônio, motivo pelo qual evoquei os tipos descritos por
Euclides da Cunha. Motivo também pelo qual o drama Os Sertões é
recorrente quando se tenta apaziguar arestas classistas num país
permeado por brutal violência e exploração de classe.}~

\emph{Mas nem haveria tanto a falar deste Antônio além dele ser mais um
Antônio perambulando por aí. Mesmo que eu acabe dando crédito ao meu
relativo sadismo ou medo de ocasião por tê"-lo cutucado com dinheiro
depois de seu exibicionismo. Quando uma nota de vinte reais virou objeto
de desconfiança e desejo e trouxe à tona uma capacidade de manipulação
típica tanto de dependentes químicos quanto de pessoas que vivem na
marginalidade.}~

\emph{Recurso esse que resumiu meu ``documentário'' imprevisto sem
câmera e foi meu momento criativo de poder extrair da mesmice miserável
das ruas uma possível pérola da brasilidade.}~

\emph{Juro que, quando tudo acabou e fui ao restaurante no Brás, pude
balizar na memória recente a junção do que tinha sido dito com o que
tinha sido insinuado nas entrelinhas.}~

\emph{Talvez por esse motivo eu diga ao leitor com bastante sinceridade
as seguintes palavras: tome você cuidado e fique muito crítico com tudo
o que se fala e escreve por aí sobre drogas, principalmente sobre o
crack que é agora a ``bola da vez'' e ocupa o lugar de honra de bode
expiatório de tantas mazelas brasileiras.}~

\emph{Afirmo tudo isso sem precisar expor os motivos sobejamente
conhecidos de que o crack é de fato perigoso e pode gerar gravíssima
dependência. Sei bem disso, mas afirmo que o crack não é nada mais do
que outra opção banal da busca de um efêmero e suspeito prazer e no
pífio caminho existencial de muitas pessoas a quem se nega o direito de
serem pessoas.}~

\emph{Daí se pode perfeitamente imaginar que esse contexto maldito de
abuso, proibição e marginalidade contribua sim para que certas pessoas
mal informadas atribuam não apenas um caráter demoníaco ao crack.
Trata"-se também de um contexto que indica um motivo mais do que apenas
``químico'' para tornar essa droga violentamente adictiva em certos
redutos.}~

\emph{Em resumo: o problema amplamente estudado do crack entre aqueles
que são examinados e bem (ou mal) tratados, ou que dificilmente são
tratados de fato, é mais do que neuroquímico, porque também é social e
cultural, senão até de linguagem e comunicação. Sem falar que a
dependência verdadeira de crack é muito difícil de tratar.}~

\emph{No entanto, me parece que para este Antônio Santiago o que
importava era a busca infinita de um sentido de vida e de uma pífia
glória nos restolhos miúdos de seu caminho brasileiro, desde a miséria
familiar na Paraíba até ele constatar outras desilusões familiares em
São Paulo.}~

\emph{Tal como acontece com tantos outros moradores de rua perambulando
nas ruas e nas várias cracolândias. Eis então, que para eles, surge a
pedra anti"-filosofal, a pedra apenas química, a pedra tentadora no
caminho, a pedra economicamente cara de ser adquirida e rapidamente
fumada que desaparece em minutos no cachimbo de várias maneiras
improvisado.}~

\emph{Antônio Santiago, enfim, não deixava de ser mais um morador de
rua, mas ele saiu de si mesmo como personagem, ele de repente
transcendeu o estereótipo do morador de rua para vestir a casaca de um
ator maior.}~

\emph{Ele o fez sem qualquer intenção prévia, assim no veio do
improviso, expressando o não dito que ecoou nas minhas palavras internas
como o dito mais verdadeiro. Por isso mesmo eu fiquei (e felizmente) sem
os meus instrumentos e as minhas referências médicas costumeiras para
entender o que se passava.}~

\emph{Eu tive de recorrer ao capricho pessoal, à minha condição
privilegiada classista e ao ato brincante até meio besta do dinheiro
para me desvencilhar de um indivíduo ou aparição e logo cair fora de um
show pequeno de rua e me refestelar no conforto e na boa comida.~}
