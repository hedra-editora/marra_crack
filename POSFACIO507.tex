\chapterspecial{Posfácio}{Droga, medicina e literatura}{}
 

Os textos que integram este livro dizem respeito a usuários intensos e,
ao menos, abusivos, de drogas psicoativas. Quase todas essas pessoas são
dependentes de crack, ou tiveram bastante contato com a droga.

As estórias (ou histórias), que para alguns leitores podem ser
assustadoras, são basicamente reais ou contêm muito pouca ficção, e
foram construídas a partir de depoimentos colhidos em atendimentos
médicos, porém não visando somente a intervenção médica.

Apenas um dos textos --- ``O canto da nóia''~\mbox{---,} surgiu de uma visita
que fiz a uma certa ``cracolândia'' da Zona Leste de São Paulo,
visita~que acabou sendo uma consulta coletiva a céu aberto.

Todas as informações referentes ao penúltimo texto (``O sobrevivente de
Tróia'') não me chegaram através de atendimentos médicos e sim através
de atividades com meus grupos de teatro.

Ao final de cada crônica, seguem,~em itálico,~depoimentos pessoais,
observações, opiniões, dados complementares, e tudo isso de maneira a
compor o que se poderia livremente considerar como os ``bastidores'' de
cada um dos textos, de maneira que o leitor possa, conforme sua escolha,
investigar mais a fundo cada estória e tomar conhecimento pleno daquilo
que constitui, na verdade, muito mais fato do que ficção.

A maioria dos textos, escritos de preferência como crônica,
conto"-crônica ou depoimento estilizado, aborda --- eventualmente com
certa objetividade porém mais comumente de maneira vaga --- um (ou mais
de um) transtorno psiquiátrico.

No entanto, e mesmo que todas as pessoas mencionadas no livro sejam
dependentes ou no mínimo abusem de drogas, por vezes não se identificam
outros transtornos além da própria adicção.

 
