\chapterspecial{Crack entre dois julgamentos}{}{}
 

Robinho\footnote{Os nomes foram trocados para preservar a identidade dos envolvidos.} tem vinte e oito anos. Ele admite haver
usado várias drogas sem ter tido maiores problemas, porém aos vinte e
seis anos conheceu o crack, viveu outra experiência que o levou ao
``fundo do poço'', e precisou de ajuda e tratamento.

Robinho tem olhar observador, analítico e levemente malicioso. Seu
discurso é claro, impositivo e um pouco formal, e se ele é simples,
educado e sintônico, deixa transparecer um envolvimento com a
malandragem e o crime.

Eu tive a impressão desse envolvimento no início da entrevista e, não
sem motivo, observo em Robinho uma agressividade velada e distante.
Então decido olhar firme para ele e arriscar um palpite intuitivo.

— Alguma coisa séria e incomum deve ter acontecido com você aos vinte e
seis anos.

Robinho me observa atentamente e comunica surpresa e concordância por
meio de expressões faciais e movimentos de cabeça. Assume uma feição
desconfiada de quem se lembra de algum fato importante e hesita em dizer
o quê.

Após nova pausa prolongada, faz"-se presente na sala de atendimento uma
súbita confiança e uma aliança temporária. Robinho sinaliza que vai
abrir o jogo para que possa continuar à vontade no tratamento.

Sim, ele admite que aconteceu uma coisa grave, e se coloca numa postura
reflexiva e tensa. Então declara que aos vinte e sete anos saiu do \versal{PCC}
por um motivo muito importante. Não devido ao crack, porque o crack foi
consequência e não causa. Sem precisar dizer que ele sabe muito bem que
os ``irmãos'' do partido não tolerariam uma droga proibida para uso,
embora livre para comercialização. Ainda mais porque Robinho não foi um
ladrão qualquer.

Ele subiu na hierarquia do partido, participou de assaltos maiores como
roubo de carga e teve um padrão de vida razoavelmente bom. Teve dois
carros, moto e casa na praia, tirou proveito da vida e esbanjou bebidas
finas e ``farinha branca'' em muitas reuniões e festas, como vinha
fazendo desde adolescente.

De repente eu percebo em Robinho uma sinceridade confessional. Que
costuma surgir naquele momento crítico em que o adicto se encontra numa
situação limite sem mais a perder e sente que precisa botar tudo para
fora. Caso contrário o sofrimento irá aumentar com recaídas sucessivas e
retomadas sem fim da busca infernal pela ``brisa'' das drogas.

— Vou contar ao senhor como se deu minha saída do partido.

\asterisc{}

Robinho nasceu e cresceu em Cidade Tiradentes, Zona Leste de São Paulo.
Foi uma criança agitada e fez de tudo que se espera quando se é moleque
na periferia. Mas bem cedo fez incursões também no que ele chama de
``coisa errada'', feito a participação em pequeno tráfico e pequenos
furtos.

Até aí, no entanto, não há grandes novidades. Nas ``quebradas'' onde
Robinho cresceu é comum a participação de muitos moleques em atividades
dessa ``coisa errada'', sem que todos sigam, mais tarde, o caminho da
contravenção ou, na condição de usuários, caiam na dependência de
drogas.

No caso de Robinho as tentações para a ``coisa errada'' aconteceram
junto a muita influência religiosa e, principalmente, evangélica, em
lugares cheios de igrejas de ``crentes'', sendo que a família de
Robinho, bastante crente, nunca foi ligada ao crime.

Esse convívio de religião e crime na periferia distante é comum. É~como
se houvesse um instável equilíbrio entre o sagrado e o profano e entre
igreja e contravenção. Embora o que acabe valendo na práxis seja
a imposição da lei crua da vida em um mundo de homens que se organizam
nas possibilidades dos seus desejos e ambições. Onde o Estado e a
cidadania estão pouco presentes. Onde alguma ``ética'' funcionalmente
prática e cruel regula a vida das pessoas a mando de uma ordem e
interesses suspeitos.

\asterisc{}

Robinho teve um amigo muito chegado chamado Vado. A~família de Vado
sempre morou na mesma vizinhança, sempre foi muito conhecida da família
de Robinho e é também evangélica.

Desde crianças Robinho e Vado frequentaram bastante a mesma igreja,
onde iam a festas e a encontros na companhia de familiares, e onde todos
cantavam louvores e usavam as melhores roupas. Mas na época da
adolescência os dois amigos andaram largados pelos becos e espaços
vazios de Cidade Tiradentes numa relativa liberdade passageira. Fizeram
traquinagens, empinaram pipas, jogaram bola e coisa e tal.

A vida tomou seu rumo conforme os caprichos de cada um e, na medida em
que surgiam nos dois uma certa inquietação e uma rebeldia,
acrescentaram"-se ainda certas influências e oportunidades.

Como sempre, a ocasião faz a hora.

Quando ambos estavam no final da adolescência, Robinho entrou no \versal{PCC} e
Vado entrou numa facção rival. Encontravam"-se uma vez ou outra e,
quando Robinho cresceu na hierarquia do partido, os ``irmãos'' sabiam
dessa amizade e faziam vistas grossas. Isso, todavia, não é de se
estranhar, porque ligações com rivais ou eventuais inimigos é comum na
história dos homens e dá um gosto adicional na dinâmica das
organizações, seja do crime organizado ou não.

Acontece que, com o passar do tempo, tudo se complicou. Os ``irmãos''
alertaram a Robinho que Vado estaria metido em ações que levaram à
morte alguns membros do partido. Robinho, por gozar de respeito e
posição, achou que exerceria influência pessoal para resolver aquele
caso e reafirmou sua vontade e determinação de que Vado fosse
respeitado porque era seu amigo de infância, família e igreja.

Os ``irmãos'' disseram a Robinho que trouxesse Vado para uma reunião
apenas de esclarecimentos. Robinho se preparou para essa missão e,
devido a desconfianças e rivalidades entre facções, criou um estratagema
para que Vado chegasse à reunião enganado e sem saber que era a pessoa
esperada.

Robinho buscou o amigo na sua casa em Cidade Tiradentes e inventou a
desculpa de que se tratava de um assunto de rotina. Garantiu que não
haveria problema algum.

\asterisc{}

O lugar da reunião era um galpão resguardado e estrategicamente bem
situado onde se faziam julgamentos. Quando Robinho chegou trazendo
Vado, estavam ali presentes vários membros do partido. Parecia um
encontro importante e era mesmo.

Depois dos informes iniciais passaram"-se às acusações. Robinho sentiu
que sua influência para intervir era praticamente nula. Seu amigo logo
percebeu que a reunião era por causa dele.

Defendeu"-se, argumentou, virou"-se como pôde. No entanto, as insinuações
e testemunhos o incriminavam, seus argumentos perdiam terreno e os dos
outros, ganhavam.

As acusações foram repetidas com vigor e agressividade verbal. Todos os
membros da reunião, colocados num semicírculo, passaram a olhar
fixamente para Vado, que ficava cada vez mais acuado. De repente
Robinho avistou três ``irmãos'' fortes se aproximando de Vado por
detrás.

Foi uma cena de fulminante rapidez. Os três ``irmãos'' tomaram de um fio
de varal e o enrolaram num só golpe no pescoço de Vado que foi erguido
bruscamente.

Suspenso do solo, Vado agonizou por alguns minutos até ser tombado como
massa inerte. A~reunião foi encerrada.

\asterisc{}

Neste momento da entrevista, os olhos de Robinho marejam e ficam meio
estatelados.

— Ele demorou pra morrer, ele demorou pra morrer. Podia ser mais rápido,
mais rápido, ele demorou pra morrer, demorou.

Robinho, mais homem do que rapaz, treme. Olha, meio confuso, para
várias direções. Eu aguardo que ele retome um pouco a calma e a
segurança.

Pouco tempo depois, Robinho assume a mesma postura do início da
entrevista e me diz, com firmeza, que, após a execução de Vado, tomou
uma decisão: pedir sua saída do partido.

Antes que eu coloque dúvidas a respeito dessa decisão, Robinho me
adianta que sabe perfeitamente ser muito difícil sair do \versal{PCC} depois que
se consegue certos postos. Não é novidade alguma quando se trata de
facções do crime organizado mundo afora.

Mas ele tem algo mais a dizer.

\asterisc

Robinho volta a falar do crack. Confessa que este uso, intenso e
contínuo, começou após a morte de Vado. Ele passou a sofrer de crises
de persecutoriedade dentro do que seria a mais perfeita ``noia''. Ouvia
``vozes'', e ainda mais enquanto tinha a impressão de que sua amizade
com Vado poderia lhe render acusações de escancarada cumplicidade.

No entanto, havia tomado uma firme decisão. E~se~por acaso seu pedido de
saída do partido chegou a lhe parecer uma forma velada de suicídio, ele
estaria pronto para tudo o que viesse.

O julgamento durou umas oito horas. Mesmo um pouco ``detonado'' pelo
crack, Robinho debateu bastante, foi ameaçado e defendeu"-se de
acusações fazendo"-se advogado de si próprio.

Sendo prudente e ardiloso, ele não fez sua defesa em nome apenas de uma
amizade. Foi ``político'' o necessário para não deixar de admitir
verdades convenientes que satisfizessem uma ética suspeita e implacável.
Mas reconheceu a amizade com Vado desde a época de adolescência, sem
deixar de citar os exemplos das famílias evangélicas e a frequência às
igrejas. Não se mostrou revoltado pela execução brutal de Vado para não
contaminar o julgamento com apelos sentimentais.

Robinho fez questão de passar a mensagem de que se fez homem pelas vias
do crime. Enfatizou respeito às leis do partido e, em momento algum, deu
a impressão de querer se colocar acima delas.

Robinho aproximou, enfim, o discurso tosco da gangue ao discurso
refinado de uma assembleia de cidadãos em alguma \emph{polis} do
planeta. Mais ainda: talvez ele tenha reproduzido alguma lição da
Política que seria endossada não por Aristóteles e sim por Maquiavel.
Embora exista, no caso de Robinho, bastante sentimento, e Maquiavel
separe os sentimentos pessoais dos interesses do Estado.

\asterisc

Robinho realmente temeu que sua saída fosse negada e houvesse terríveis
consequências. Temeu pela vida quando sua mente o fazia ver toda hora
Vado erguido pelo fio de varal e se contorcendo suspenso no ar diante
de todos os ``irmãos''. Por isso ia se mantendo pessimista, mas não
perdia a fibra e continuava na sua defesa.

De repente eu arrisco uma observação delicada:

— Se você tivesse dito que estava usando crack teria morrido, não é
verdade?

— Sim, porque isso não é aceito no mundo do crime e ainda mais na minha
posição.

\asterisc{}

Quando o julgamento terminou, o ``irmão'' graduado que presidia a
reunião proferiu a sentença.

Rezava a sentença: ``você não é mais irmão; pode fazer o que quiser; se
quiser roubar pode roubar, se quiser ser trabalhador pode ser
trabalhador; se quiser ser zé-povinho pode ser zé-povinho''.

\asterisc{}

Mais aliviado agora, Robinho volta a me falar sobre o crack.

A dependência seguiu rápida porque ele já era usuário de ``farinha
branca''. O~crack, porém, assumiu uma proporção diferente. Não houve
descobertas novas e ele não passou pela fase do prazer impactante e nem
sequer pela busca do prazer.

O crack foi um recurso compulsivo de alívio inútil. O~sentimento de
culpa não o deixava e a família evangélica de Vado sempre acusava
Robinho de ser o agente do \versal{PCC} que foi buscar seu amigo em casa, o
enganou e o levou ao matadouro.

Incapaz de rebater a críticas que continham um fundo de verdade,
Robinho afundou na culpa e acabou se mudando de bairro. Mas o crack o
acompanhou e tomou conta dele. A~pedra passou a ser um recurso maldito,
obsessivo e repetitivo para aliviar uma tensão interna crescente. Seu
``fundo do poço'' ampliou"-se num torvelinho de autoincriminações.

\asterisc{}

Ao final da entrevista não me ocorre qualquer história mitológica que
esclareça o caso, e sim a lembrança de uma frase de Kierkegaard, um dos
precursores do existencialismo: ``querer ser quem se é realmente é na
verdade o oposto do desespero''.

Pois então, uma chave neste caso é fazer com que o desespero de Robinho
o auxilie a encontrar seu verdadeiro ``eu''. Assim, de uma tal maneira,
que essa chave esteja situada na passagem pelo caminho da pedra concreta
dolorosamente sedutora; mas também esteja situada a meio caminho de
outra pedra simbólica lhe despencando, esmagadora, na consciência e
diante da qual as pedras menores vendidas nas biqueiras constituem
apenas fumaça.

Ao final da entrevista Robinho volta a ter a mesma postura do início.
Momentos de calma e silêncio sinalizam uma mensagem implícita de que ele
não teria falado nada. Ou queria determinar um certo apagamento, ou um
esquecimento de tudo.

Ele me agradece e afirma que prosseguirá no tratamento. Retira"-se ---
educado, melancólico, algo formal e com um ar de ligeireza e malícia ---
equilibrando"-se com alguma força física e psíquica que ainda lhe restam.

Num sentido bom e mau eu percebo que ele é mais um obsessivo nesta vida
torta e certa, e que seu desespero lhe rende esperanças de um encontro
consigo mesmo.

Torço para que Kierkegaard tenha razão. E~quanto a Robinho, ele nem
precisaria saber agora da existência desse filósofo, que era obcecado
com o tema bíblico do sacrifício e também se debateu entre outros tipos
de~desespero.

\asterisc{}
%\begingroup\small

\emph{Quando trabalhei na área da Nitroquímica, em
São Miguel Paulista, encontrei comunidades com cinquenta por cento das
pessoas vinculadas ao narcotráfico, seja diretamente, ou então
indiretamente como colaboradores.}

\emph{Tratava"-se de uma rede economicamente ativa girando em torno do
negócio da droga ilícita. Nesse meio e nesse mesmo espaço cruzavam"-se
chefes do tráfico, bandidos perversos, bandidos com caráter,
``aviõezinhos'', pobres diabos, donas de casa, receptadores, olheiros,
zé-povinho e o escambau a quatro.}

\emph{Como discernir aqueles que teriam transtorno
de conduta ou coisa parecida, aqueles supostamente considerados
``sociopatas'', aqueles verdadeiramente perversos, ou aqueles
simplesmente habituados a viver em determinado modo de produção e até
muito em paz com suas consciências?}


\emph{Bem ou mal nos becos da existência, há problemas em todo canto e
há muitas soluções apenas parciais. No palco do mundo real todos se
viram à própria sorte mesmo que, segundo os crentes, tudo esteja
entregue nas mãos de Deus.}

\emph{E tem mais: falando"-se de uma população jovem ou jovem adulta e
vulnerável para o crime e para as drogas pesadas, existe um território
intermediário dos que --- principalmente adolescentes periféricos ---
passam por períodos críticos de delinquência e uso de drogas e depois se
tornam, aos trancos e barrancos e sem terapia afora as igrejas, ah, se
tornam uns trabalhadores ou peões, engrossando essa massa humana vivendo
longe do centro da cidade e enchendo todos os dias o trem, o metrô e as
lotações.}

\emph{Mas Robinho é, digamos assim, uma criatura distante de ser um
peão comum. Ele é verdadeiramente um ``personagem'' desta crônica e
parece ter um diferencial no meio em que habita.}

\emph{Não foi à toa que observei nele uma ``garra'', uma inquietação
produtiva, um caráter bem presente, um dilema existencial e uma
propensão a depositar confiança e a fazer alianças.}

\emph{E eu assim o digo sem que eu o considere --- diga"-se de passagem
--- como sendo qualquer modelo de virtude ou ``flor que se cheire''.
Também sou sincero em admitir que não endosso o que me foi dito por ele
ao pé da letra, apenas me deixei levar por sua narrativa clara e
sintônica.}

\emph{Além disso suponho que se tudo não é propriamente vero na íntegra,
é pelo menos} ben trovato, \emph{como reza o provérbio italiano. Mas como tenho
conhecimento de histórias mirabolantes envolvendo facções criminosas, e
como nessa área há tantas exceções quanto regras, tudo acaba sendo
possível debaixo do sol e principalmente debaixo das sombras.}

\emph{Isto é, tudo acaba sendo possível desde que tudo seja dito de
maneira convincente entre quatro paredes num desabafo, considerando"-se que
todo mundo, sem exceção, mente e diz a verdade o tempo todo e que cada
um de nós tem seu autoengano e você, leitor, não é exceção.}

\emph{Quanto às verdades ocultas em relativo segredo ou situadas por
detrás da opinião ou da} doxa \emph{comum, nós sabemos que elas existem mas
estão sempre fugindo. Eis aí, por sinal, o elemento invisível presente
na penumbra inevitável dos discursos, tal como é discretamente mostrado,
aliás, num belo conto de Edgar Allan Poe,} O Homem da Multidão (Man of
the crowd), \emph{em que bem no início Poe escreve uma frase em alemão ---} es
lässt sich nicht lesen, \emph{ou seja, é aquilo que não se permite ser lido, e
que poderia ser também aquilo que (talvez) não se permita ser dito.}

\emph{Como está numa frase de Dante aludindo às profundezas do que
transigiria a consciência a partir de um certo limite. Uma frase citada
bastante por mim:} s'i'odo il vero, senza tema d'infamia ti rispondo \emph{(se
eu escuto o verdadeiro sem receio da infâmia eu te respondo) --- uma
referência literária ao ponto máximo de tensão em que se dá uma implosão
do autoengano e vem tudo para fora, ou seja, chega a hora da verdade,
chega um momento catártico.}

\emph{Mas enquanto não se chega a este ponto, a realidade pode ficar
mascarada naquilo que não se permite ser dito ou ser lido. Daí a
necessidade de uma ``escuta'' verdadeira, íntima, que rompa com a
retórica vulgar do autoengano ou do discurso que é lugar comum.}

\emph{Nada mais próximo da saga terrível de Robinho, que eu diria ser
quase tragédia grega a partir de um percurso canhestro de um anti"-herói
das sombras a purgar, pelo sofrimento do crack, a culpa relativa pela
morte do amigo Vado, misturando sentimento legítimo com modestas
maquinações maquiavélicas numa assembleia de contraventores.}

\emph{Quero aproveitar a oportunidade para comentar que Robinho lembrou
para mim muitos atendimentos feitos com meninos do crime sem qualquer}
status \emph{elevado dentro das facções. Eles eram, na verdade, uns ``pés de
chinelo'', uns aventureiros meio doidinhos, uns zés-ninguéns da vida.}

\emph{Muitos acabavam mortos num rolo compressor da ``limpeza''
promovida pelos que realmente fazem carreira profissional na
bandidagem.}

\emph{Eu me lembro de ter conversado bastante, há alguns anos atrás, em
Ermelino~Matarazzo, com um deles, adolescente bem dotado de cabeça e que
também teve influência evangélica marcante.}

\emph{De vez em quando eu o chamava para almoçar comigo num boteco
próximo ao meu trabalho, e dessa maneira pude conhecê"-lo melhor. Percebi
logo que ele tinha pretensões toscamente românticas e equivocadas, e uma
vez me disse que roubava porque ``tudo era culpa do capitalismo''.}

\emph{Achei no mínimo esse discurso notável e ele me falou ainda em
socialismo. Puxa vida! Mas verifiquei depois que ele, infelizmente, nada
sabia sobre socialismo e nem sobre capitalismo. Ele era apenas um
retórico malandro. Simpático, envolvente.}

\emph{Mas ainda assim eu disse a ele que o crime, na minha opinião, é
basicamente capitalista e não busca a distribuição dos bens, muito pelo
contrário. Ele me disse cinicamente que roubava} playboys \emph{e não
trabalhadores.}

\emph{Eu respondi que normalmente não está escrito na cara de ninguém
quem é} playboy \emph{e quem é trabalhador. E~perguntei: ah, e você o que é?!
Ele não respondeu e eu resumi minha opinião lhe dizendo que ele roubava
a quem ele achava que era} playboy \emph{porque, lá no fundo, tinha vontade de
ser} playboy.

\emph{Em seguida acrescentei que manos e} playboys \emph{podem ser, além de
estereótipos, farinha do mesmo saco no imaginário de manos e de} playboys
\emph{e de toda uma sociedade que inventa manos e} playboys.

\emph{O~moleque teve um momento feliz de autocrítica e acabou
reconhecendo isso como~uma grande verdade. Desabafou admitindo que
invejava os que ele tinha em conta como} playboys.

\emph{Eu o aconselhei a não migrar para o crime organizado porque ele
não teria vocação para isso (ou sei lá se ele ainda teria mais
vocação!), esperando eu que ele fosse aprender melhor sobre o
capitalismo e o socialismo e sepultar seus devaneios imaturos.}

\emph{Mas tudo isso, caro leitor, seria quase outra história, não é
mesmo? E voltando propriamente ao assunto, veja bem: Robinho, nos dias
de hoje, não tem praticamente nada em comum com esse moleque, embora no
início talvez eles se assemelhassem; Robinho já passou muito além da
fase da inveja por um suposto} playboy \emph{imaginário e ademais não está
preocupado se o crime organizado é ou não capitalista.}

\emph{Robinho é ardiloso, moderno e global e sabe que em todo este
planeta Terra as pessoas são, tantas vezes, suas próprias aparências ou
são} personas \emph{aos olhos dos outros. E~os bandidos, como também a polícia,
podem se orientar pelas aparências, e todos são influenciáveis para o
convencimento desde que haja encenação e retórica suficientes.}

\emph{Também as tais vítimas, sejam boazinhas ou não, e também os tais
algozes, sejam monstruosos ou não, todos são ligados entre si por
estranhos laços, e Robinho tem disso uma boa percepção intuitiva.}

\emph{Enfim, volto a olhar melhor para Robinho que, na verdade,
jogou"-se mesmo no crime, fez um pacto, foi ``bom aluno'' da contravenção
e absorveu leis e normas de uma organização do submundo.}

\emph{Mas acontece que Robinho levou embutido no seu pacto dilemas
morais urdidos, mesmo precariamente, sob uma égide evangélica de
periferia. E~digo aqui, bem a propósito, e de passagem, que observo
muito dessa influência religiosa nos contraventores e nos drogados
pesados.}

\emph{Eis porque me veio à mente uma frase linda de Kierkegaard e não um
mito grego para melhor resumir o caso Robinho. Porque Robinho me
parece, de longe ou por meio de um simulacro, um ser pequeninamente
faustiano sofrendo dilacerações entre seu suposto ``eu'' mais verdadeiro
e, por outro lado, as farsas postiças de ``eus'' espúrios mediados pelas
ilusões diabólicas do crime e da droga. Daí sua grande dor e sua
relativa grandeza anônima, justamente aquilo que me empolgou.}

\emph{Seria mais ou menos isso, caro leitor, o que eu teria de comentar,
mas vou terminar agora com algumas considerações, talvez um pouco do
domínio das neurociências, para dar um fecho nas minhas apreciações sobre
o caso Robinho e o de tantos outros ``personagens'' notáveis das
sombras.}

\emph{Quando eu atendia a vários moleques do crime, percebi claramente
que eles passavam por fases nítidas que podem até ser sequenciadas
conforme uma metodologia científica.}

\emph{Eu verificava o seguinte: havia a antecipação do prazer antes do
assalto, depois havia uma inquietação frente a um desafio proposto,
depois uma tempestade de perturbação prazerosa ou de busca de prazer
envolvendo perigo na iminência do assalto, e por fim havia um período
rebote de depressão após o assalto.}

\emph{Depois se seguia um período letárgico ou refratário no qual o
moleque ficava meio amuado para que então lhe viesse uma espécie de
``coceira'' ou de vontade de pegar a arma de novo e sair por aí para
fazer uma ``fita'', sozinho ou acompanhado.}

\emph{O~mais significativo é que quase sempre o ritual valia mais do que
o próprio produto do roubo. Melhor dizendo: interessava mais, no ato de
contravenção, o meio e não o fim; e a própria busca desencadeava um
mecanismo de autorrenovação.}

\emph{De maneira bem reveladora, isso tudo aproxima"-se de um ritual
envolvendo, por exemplo, o uso de uma pedra de crack, ou de cocaína, ou
de outra droga psicoativa pesada.}

\emph{Observe"-se que essa sequência de etapas pode perfeitamente ter uma
expressão neuroquímica no tal sistema de recompensa do cérebro e seguir
no curso das mesmas vias e dos mesmos neurotransmissores envolvidos.}

\emph{Costumo dizer que há pessoas --- meninos e meninas, rapazes e
moças --- trazendo facilitações genéticas e ``feridas'' existenciais de
criação que os podem conduzir atrás das mais diversas ``brisas''
enquanto eles buscam experiências que se contraponham à monotonia da
vida rotineira.}

\emph{Sejam eles os tais ``buscadores de novidade'' ou os que preferem a
inércia de ``mamar'' na garrafa, pipar no cachimbo ou tragar no baseado.
Ambos os grupos, no entanto, acabam atolados na compulsão para a
repetição, que é na verdade uma monotonia.}

\emph{Lembro ainda que o meio ambiente pode ser decisivo para ativar
mecanismos predisponentes ou para não os ativar, direcionando
potencialidades psíquicas de hábito ambivalentes.}

\emph{Para quem trabalha com adictos, a grande questão é como tirar bom
proveito terapêutico dessa inquietação que os drogados costumam ter por
novidades. A~difícil questão, todavia, é como direcionar positivamente
essa insatisfação contínua.}

\emph{Robinho, em resumo, é um ser complexo e inacabado. Mas ele é,
afinal de contas, um homem feito, se bem que ele seja ainda um pouco} 
puer \emph{, ou é um rapaz meio menino, ou foi um menino criado numa ética
religiosa, um menino inquieto e ousado mas com a peculiaridade de estar
aberto a um olhar sobre si mesmo mediado pelo que lhe resta de
caráter.}

\emph{A~crônica e/ou o conto"-crônica pode vir dessa tensão que me
despencou, tão confessional e aparentemente tão sincera, durante uma
longa e decisiva entrevista e no bojo de uma crise de consciência, e
seja até por uma ótica kierkegaardiana do desespero.}

\emph{Digo desespero entre dois julgamentos com a pedra de crack no
meio. Ou digo desespero mediante as vicissitudes de um simulacro do mito
do herói conforme alusão distante aos ``clássicos'' modelos do teatro
grego.}

\emph{Seria isso e muito mais sobre este Robinho, caro leitor. Porque
ninguém é transparente e se Robinho é um só ele pode ser muitos!}

\emph{Até mesmo legião.~}
%\endgroup
 

 

 
