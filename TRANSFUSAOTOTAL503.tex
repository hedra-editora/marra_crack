\chapterspecial{Transfusão total}{}{}
 

A princípio seria uma síndrome do vampiro. Não distante dessa onda de
vampirismo na mídia em uma literatura \emph{trash}. Conforme vejo
bastante no metrô, no trem e nas lotações --- meninos e meninas meio
hipnotizados deleitando"-se em apelos românticos por novas versões de
Dráculas midiáticos; conectando"-se no desejo de se vampirizarem
virtualmente via \emph{net}, \emph{facebook} e coisa e tal.

É o que de imediato me vem à lembrança quando me pedem para atender um
rapaz com atitudes estranhas ou sinistras, em um local de tratamento a
dependentes de drogas.

Ele estaria proclamando um insano desejo de sangrar a si mesmo e beber
sangue alheio. Despertando uma onda de pânico ou de horror em plena luz
do dia.

Todos aqui temem que ele pegue algum instrumento de corte e se fira, ou
fira alguém para satisfazer o que ainda não se sabe ser desvairada
loucura ou fantasia mórbida de estranhíssima ``noia''.

O que se segue é uma esquiva nervosa e atordoada de pessoas e um
movimento apressado de seguranças.

Mas, de repente, eu vejo melhor este indivíduo, num instante de pausa e
silêncio.

Ele é uma criatura chamada Sidnei que, em poucos instantes, é conduzido
a uma sala de atendimento sem maiores dificuldades.

Ele mantém um olhar meio parado. Está, porém, orientado e, dentro em
pouco, responde perguntas, já conversa.

Não esconde o cínico desejo de beber sangue e de fazê"-lo jorrar de seu
corpo. E~ainda parece um pouco perdido e confuso no assunto sangue
quando levanta a camisa e expõe vários cortes paralelos de faca no tórax
e abdômen.

É um rapaz de 22 anos, vindo de Sergipe, com sotaque forte do Norte. O~tom escurecido da pele denota uma negritude mascarada. Veste"-se com
neutra simplicidade. No mais é como se fosse um peão de obra.

Quando eu pergunto a respeito do hábito de usar armas brancas, ele
entorta a cabeça para baixo. Com dissimulada calma me diz já ter furado
o corpo de algum próximo por questão de brava desavença. Mas nunca
apagou ninguém, só feriu em brigas ao ter sido provocado.

Eu tenho a impressão de que ele não seja propriamente violento. Não
parece. É~apenas um tipo próximo ao sertanejo, um matuto, entocado em si
mesmo naquela desconfiança de ficar ``na dele'', introvertido e à
espreita.

\asterisc{}

Neste encontro começa a surgir uma pequena história de vida balizada pela
racionalidade linear do tempo, do tempo que condensa breves relatos e
traz supostos fatos ocorridos ao longo de uma vida. Quando Sidnei, por
via de nebulosas circunstâncias de criação, tornou"-se progressivamente o
que se considera um dependente grave de drogas.

Ao longo de uma \emph{via crucis} de experimentos com drogas, ele
afundou no atoleiro do crack e dele se encantou radicalmente, enquanto
foi se mudando aos trancos e barrancos para São Paulo com os pais e os
irmãos e, aos poucos, desligou"-se primeiro de empregos regulares e,
depois, de ``bicos'' variados e inconstantes.

Num momento reflexivo ele admite que, caso um ``vício'' não o tivesse
jogado para o que ele chama de ``fundo do poço'', estaria preso à rotina
de uma família e de um emprego comum, indo e vindo todo dia de casa para
o trabalho.

Mas sendo ou não um trabalhador, aqui na periferia de São Paulo ele tem
até um jeito de ``mano'', como muitos que engrossam a população bastante
jovem de Guaianases. Por outro lado, ele não é uma pessoa comum e se
diferencia bastante de outros ``manos'' e dos consumidores dessa onda
\emph{trash} de literatura vampiresca.

Não apenas por causa da sua peculiar volúpia de sangue, mas também pelos
vários cortes espalhados pelo corpo --- coisa até sofisticada e cheia de
símbolos ricos. Que no embalo da fissura pelo crack seria manifestação
enviesada de um desejo forte pedindo um ritual entre autofágico e
antropofágico.

Sidnei, ao contrário de outros dependentes de crack costumeiros e
banais, é um contraventor ao mutilar seu próprio território corporal, onde ele testa e concretiza bizarras fantasias cortantes.

Não se contentando com a vontade de chupar o sangue alheio, deseja
chupar o dele próprio e expelí"-lo como num ato masturbatório de singular
prazer.

Dele brotaria um desejo vindo lá do fundo, para sair de si próprio e
também habitar os mistérios de um outro ser com quem buscasse parceria.
Ou com quem se dispusesse a trocas realizadas na agonia erótica de um
dar e receber.

Mas até aqui Sidnei parecia afinado com a proposta de outros dependentes
de crack, os quais se sentem impregnados de algum mal, de alguma sujeira
profunda a circular no sangue. Daí se voltam, quase todos, obsessivos em
busca de uma limpeza. Que aliás tem nome: palavra higienista e
marqueteira e assunto já midiático às escâncaras. É~a tal
desintoxicação, termo comum nas tais clínicas de (suposta e suspeita)
recuperação.

Sidnei, porém, acrescenta aí uma variante macabra e até original, embora
ele parta da mesma agonia de outros dependentes verdadeiros de crack;
porque esses outros costumam se sentir vagamente culpados e impregnados
do mal da droga e querem se purgar, querem apagar tanto a culpa da alma
quanto querem retirar o mal do sangue e do corpo.

E seguem no mesmo embalo das inúmeras tolices que a gente escuta todo
dia a respeito de drogas. Quando tantos dos que são chamados adictos
insistem em ficar meses a fio tomando remédios inúteis para se
desintoxicarem e se limparem continuamente.

 

Acreditando que invisíveis manchas no sangue durem muitos anos a fio;
chegando a jurar que uma terrível sujeira do sangue possa ser curada
somente com a morte ou, sendo mais do que eterna enquanto dura,
transfira"-se para meandros do além mundo.

Sidnei, todavia, deseja ser radical, e carrega a insistência ousada de
quem esteja querendo, mal e porcamente, uma transfusão total.

Justamente ao confessar um plano de transcendência inusitada na medida
em que busca jorrar"-se de dentro de si mesmo; seguindo além da vulgar
fantasia vampiresca de apenas sugar o sangue alheio; perdendo"-se nos
embalos de uma ``noia'' mais profunda e com pretensões maiores do que
mordidas convencionais de inúmeros Dráculas midiáticos.

\asterisc{}

O ambiente aqui no meu local de trabalho é agora de calmaria. E~se as
pessoas já estão aliviadas do velho medo de sangue, digo calmamente a
todos: este rapaz nordestino vindo de Sergipe é tão somente mais um a
ser tratado.

Se bem que eu gostaria de dizer também a ele do quanto eu o considero
vítima das contradições do Brasil moderno: --- ser"-humano"-droga"-alienado
em sociedade líquida de pessoas"-mercadorias, transeunte"-transitório em
busca de alternativo desejo.

Por um líquido motivo ele me lembra aquele personagem"-trabalhador do
filme ``O homem que virou suco'', rodado em plena ditadura militar. Um
personagem que renasce hoje como um outro personagem talvez um pouco
semelhante, igualmente frágil, aventurando"-se num Brasil mais
democrático.

Nesse caso, sendo aquele que segue no caminho da pedra incandescente, em
bizarro idílio entre atos concretos e virtuais. No embalo moderno do
fetiche da pedra, ou no rumo de um passaporte simbólico para domínios de
um inefável terror.

Mas ele insiste em se refazer da sujeira profunda para atingir a
cristalina pureza. E~na busca pela transfusão total, ele deseja realizar
atos íntimos não de maneira eroticamente costumeira, porém eroticamente
enviesada, com outras criaturas ditas humanas.

Isso ele procura concretizar desde o profundo limiar de um alternativo
desejo por gozo, até chegar a pretensão de um troca"-troca. Para dar e
receber, para eliminar uma velha carga e depois recolocar de volta uma
nova, e assim atingir um nível desejado de cura pela transfusão radical.

Desde um Sidnei pessoal até se fundir entre sidneis impessoais com suas
almas agônicas e corpos cheios de bizarra volúpia. Numa condição do que
se diria grave transtorno \emph{borderline} da personalidade, que faz
com que anjos se confundam com demônios enquanto demônios se fazem de
anjos. Em situação dita limítrofe e crepuscular.

\asterisc{}

Mas agora tudo aqui está sob relativo controle.

Sidnei já é um simples paciente preparando"-se para ir embora e depois
retornar. Sem dizer que ele é também número de prontuário. Seu caso
nebuloso se desvanece perante um mero e pontual dado estatístico no que
se assume como diagnóstico.

\asterisc{}

Estou de volta ao trem rumo ao centro da cidade. Vejo meninas e meninos
em leituras e visões de belos corpos adolescentes, no erotismo virtual
de trocas de sangue. Em território seguro onde Sidnei peão nordestino e
brasileiro comum destoaria por ter vindo de caminhos diferentes.
Descolado, via louca radicalidade, dessa moda vampiresca de literatura
\emph{trash}.

Quem sabe desejando, no seu beco de solidão, criar um autofágico romance
solitário de si próprio enquanto busca sugar"-se e devorar outro ser
humano vago e desejado nos limites infinitos da ``brisa'' da pedra
incandescente de cocaína.

\begin{center}\asterisc{}\end{center}
\begingroup\small

\emph{Os temas vampirescos são prolíficos, riquíssimos e polissêmicos.
No caso do crack e da suposta epidemia que, conforme dizem por aí,
assolaria o país, raramente surge melhor instrumento simbólico de
avaliação do que a velha história da ``picada do vampiro''.}~

\emph{Sim, porque a metáfora sintetiza os horrores criados em cima dessa
figura misteriosa e mal compreendida que é o adicto em crack, vulgo
``noia'' retratado habitualmente como um zumbi caricato, magrelo, de
rosto chupado e com jeito de psicopata de filme classe C.}~

\emph{No entanto, simbolicamente falando, o tema folclórico e
arquetípico do vampiro --- surgido antes dos Dráculas bem conhecidos e
sendo um tema tão antigo quanto a memória da humanidade --- expressa
clareza sobre o mundo das sombras infernais que, segundo muita gente
acredita, é a sanha maldita do crack e de seus eternos dependentes.}~

\emph{Não é preciso explorar muito este tema de forma médica ou
científica para demolir algumas crenças vãs, como a de que um uso único
do crack lança o coitado do usuário na eterna dependência --- tola crença
aliás bastante comum.}~

\emph{É preciso mostrar também o medo atávico do contágio transcendendo
o que se tem ou o que se imagina como corpo. Contágio da suposta alma que veicula uma semente do mal acreditada perene.}~

\emph{Mas eu sei que nesta onda perigosa do crack há que se considerar,
é claro, os problemas seríssimos da adicção pesada e relativamente
rápida ocorrendo em muitos mas não em todos os usuários.}~

\emph{Não há como recusar também as considerações objetivas e
científicas da questão. Isso é evidente. No entanto, há que se
considerar o imaginário popular constelado na ``picada do vampiro''.}~

\emph{É por isso que aquele ser humano pobre diabo que entrou em cena em
um centro~de atendimento público apavorando a todos com reais e
promíscuas ameaças de trocas de sangue, ah sim, é por isso que ele, com
razão, semeou um pânico passageiro.}~

\emph{Fomos pegos de surpresa e por esse medo inconsciente! A nós que
estamos acostumados com tipos enlouquecidos e possuídos, com surtos
psicóticos e o escambau.}~

\emph{Eu me lembro que ele trazia uma horrenda delicadeza transbordando
em sua líquida e rubra loucura. Afinal de contas, ele pode ser reduzido,
muito provavelmente, a uma condição tida na área psi como personalidade
limítrofe, transtorno} borderline \emph{grave.}~

\emph{Ele reúne sintomas e sinais inequívocos de uma entidade mista,
ambivalente, desafiante e complexa. Entidade que evoca as psiques
fragmentadas que não têm aquilo que se chama de} self\emph{; ainda mais ele,
como tantos outros, que são personalidades seduzidas pelas sombras e
trocam o luminoso pelo sombrio. Que são tipos humanos eroticamente
ávidos de contatos bizarros e buscam, até inconscientemente e via
regressões animalescas, um sugar obsessivo de fase oral.}~

\emph{No caso deste Sidnei tudo se constelou em um personagem real,
concreto e vivíssimo a trazer cinicamente uma proposta de transfusão
total, como ele calmamente deu a entender na entrevista, enquanto
oscilava nos limites frouxos da sua personalidade.}~

\emph{Eis aí, portanto, uma condição complexa habitualmente encontrada
em adictos muito graves, para os quais a chamada droga --- crack ou outra
--- é uma moeda de troca para uma viagem erótica maldita, recheada de
lúgubres tinturas românticas e fazendo eco ao que tem sido moda
passageira na literatura} trash \emph{sobre vampiros.}~

\emph{Literatura feita principalmente para adolescentes mirarem corpos
esbeltos e esculturais em meio a um descarado desejo de sexo, de
contato, porém apenas ampliando uma visão medíocre, que é reduzida a
flertes nas redes sociais.}~

\emph{Então volto ao início quando me apareceu este Sidnei tão anônimo
que, dentre muitos outros usuários também portadores de grave transtorno
limítrofe da personalidade, foi quem melhor sintetizou um romantismo
satânico que essa literatura} trash \emph{não consegue fazer por ser
pasteurizada demais.}~

\emph{Admito, enfim, ter viajado um pouco além nas minhas conclusões a
respeito desta criatura atormentada, e digo francamente que esta
criatura deveria estar sendo movida por uma obsessão de cura, por uma
insistência em juntar seus cacos, seus pedaços.}~

\emph{Da mesma forma como os vampiros literários, e os vampiros
clássicos, e os vampiros folclóricos ou também os midiáticos são movidos
pela ânsia de cura da morte, ao sugarem eternamente o sangue e a alma
dos vivos.}~

\emph{Para quem se disponha a fazer uma viagem mais profunda nas
contradições inimagináveis da dependência pesada de crack (e de outras
drogas também) é muito importante se reportar a uma ânsia profunda que
mistura erotismo das sombras com religiosidade sincrética.}~

\emph{Sem falar dos polissignificados da sexualidade humana como se os
mesmos significados compusessem um bizarro e descarado poema concreto,
como se pudessem ser escritos em gestos cortantes e perfurantes, em
rasgos suculentos na carne viva, em rasgos cinicamente gozosos e
perpetrados até com uma elegância de um Marquês de Sade. E~ainda mais
dentro desta onda, cuja semiótica química --- eu diria, tomando
liberdades --- precisa ser melhor decodificada nos descaminhos
caprichosos das sombras iluminadas, ou das iluminações sombrias.}~

\emph{Mas acontece que depois da tempestade vem a bonança, e foi o que
se deu quando aquele rapaz recebeu seu rótulo, recebeu seu diagnóstico e
apaziguou os circunstantes com as bênçãos da medicina ou da saúde
pública.}~

\emph{Mal sabendo ele de sua condição limítrofe e oscilante, ou sabendo
sem o saber. Porque depois, caro leitor assustado, eu não me lembro de
tê"-lo visto mais.}~

\emph{Acho que ele não voltou e não sei se ele seguiu seu caminho
insistindo em transfundir"-se, ou se promoveu sustos do tipo filme
``sexta"-feira treze''.}~

\emph{Sem dizer ainda que muitas personalidades limítrofes querem juntar
suas partes em busca de um} self \emph{e evocam ao mesmo tempo susto, desafio,
obsessão, surpresa e tormento. Ou tudo isso junto. Isto é, quando se
juntam se desjuntando.}
\endgroup
 

