\chapterspecial{Quatro tempos antes de um acerto de contas}{}{}
 

\section{Primeiro tempo}

O senhor procure me entender. Parece filme de terror: minha mãe bebendo,
depois caída no chão, ouvindo vozes, tendo visões, gritando, berrando;
meu irmão me batendo e eu sem saber porque; e meu pai, uma sombra que um
dia passou.

\asterisc{}

Comecei a usar droga com doze anos. Usei de tudo e aos quinze caí no
crack. Na época eu traficava, roubava e costumava parar muito na Febem.

Confesso dois homicídios. O~primeiro com dezoito anos quando fui
enganado por um comparsa na partilha de um assalto.

Se tive remorso? Não. Segui a lei do crime. Só fiquei assustado nos
primeiros dias com medo de me caguetarem.

O outro homicídio foi por causa de um olhar atravessado dentro de um
bar. Eu falei: --- qual é a tua assim de me encarar?! O cara disse que
eu era folgado. Rolei com ele no chão, puxei uma faca e rasguei a
barriga do cara.

Se me arrependo? Sim. Pedi perdão a Deus, mas ainda vejo um corpo na
minha frente e escuto a voz do morto.

Em toda minha vida teve muito assalto a mão armada, dito 157. Só uma vez
eu e meus comparsas vacilamos quando a gente foi surpreendido na manha
por policiais à paisana.

Então conversei com os comparsas na viatura: nada de entregar o \versal{B.O.},
mano.

Os gambés me deixaram nu, deram choque no meu pau, deram soco amortecido
pra não deixar marca, me penduravam de ponta cabeça, depois desviravam e
aí começava tudo de novo.

De madrugada os companheiros de cela ameaçaram dar voz de rebelião.
Enquanto os gambés me detonavam os outros detentos berravam. Eu via a
morte. Até que chegou o delegado da manhã, ficou preocupado com o clima
no \versal{DP} e ordenou o fim da tortura.

O delegado me estendeu papel e caneta. Minha mente turvava. Assinei
cambaleando porte ilegal de arma. Nenhum 157. Caguetar não ia não.

Minha história, desde moleque, está muito ligada no crime e na droga,
principalmente no crack. Mas eu sempre fui respeitado quando os
comparsas me chamavam pra dar cobertura e aparecia muito assalto. Tudo
era adrenalina e ao mesmo tempo meio de vida, né!?

Eu também brinquei, senhor, de ser capitalista quando fiquei sócio de
uma biqueira. Eu vivia da pedra e para a pedra! Cheguei a ser consumidor
de mim mesmo --- tá ligado?! --- girando entre a minha brisa e o meu
negócio.

Eu estava até meio no sossego quando meu sócio ainda me chamou pra uma~correria 
da hora porque sabe que sou bom de serviço, e me tentou em mais
uma aventura de ação e adrenalina.

De repente eu estava pra voltar aos velhos tempos de 157. Mas mudei de
ideia e disse: escuta aqui, meu irmão, não quero mais saber desse tipo
de crime, e peço proteção a Deus. Ele estranhou. Perguntou se eu ainda
estava firmeza na biqueira que a gente tinha juntos. Eu disse que estava
só meio firmeza porque tinha tomado uma decisão: passar minha parte
adiante. Ele estranhou mais. Eu admiti que tinha acontecido comigo uma
crise de consciência, e daí comecei a ir na igreja onde falam que o
maligno sempre ronda por perto.

O senhor quer saber? Eu quero outra coisa na vida, ainda que eu sempre
me considere guerreiro --- de fé como nas letras dos Racionais.

Embaçado explicar, né? É que passei a vida no crime, apesar de sentir
fome de justiça, um pouco torta pra dizer a verdade, mas ela pode ser
simples e direta. Não me conformo vendo uma pessoa sofrer na minha
frente e quero ajudar, mesmo fazendo outra pessoa sofrer.

O senhor pode acreditar: já pensei em ser justiceiro, até maldito, e por
que não?

Também aprecio paz e alegria. Posso ser cruel mas tem hora em que sou
criança. Viajo solto nos meus pensamentos, bem brisado e de boa curtindo
uma paranga suave na minha quebrada pra relaxar.

 

\section{Segundo tempo}

 

Confesso um medo de morrer, menos pela vida na droga do que pela droga
de vida. E~tenho pela frente um acerto de contas que desejo esclarecer.

O senhor sabe que tudo neste mundo tem história, mas o que me interessa
é a história da biqueira, que funcionava onde tinha um bar, existia
antes de mim, vai continuar existindo, teve outros donos, e vem passando
de mão em mão.

De repente eu me vi meio dono, e meu comparsa se viu assim, só que nós
estamos passando o ponto porque o mundo gira e os acontecimentos também
giram e são negócios, né?

O problema são as tretas que podem surgir em todos os negócios, pois o
planeta dá suas voltas e os irmãos --- de sangue ou não --- se encontram
e se desencontram na mão e na contramão do mundo.

E aqui começa na verdade a história.

O meu parceiro atual sabe que eu fazia umas correrias em sociedade com o
irmão dele. Daí surgiu o problema do dinheiro daquele assalto que a
gente fez, os três juntos, o meu último 157.

Eu investi o dinheiro todo do assalto na biqueira junto com este meu
parceiro atual. Acontece que o outro, o irmão dele, está preso.

Ele ficou desconfiado e mandou umas pipas pra se informar e me
pressionar. Eu consegui fazer a cabeça do meu parceiro pra ele segurar
as pontas até o irmão dele sair da cadeia e a gente acertar o que tem de
acertar.

Então surgiu outra treta com o irmão do rapaz que está na negociação da
biqueira, e que é da outra rua na mesma vila onde eu moro.

Mas aí a discórdia começou num momento em que a danação correu solta a
troco de besteira.

Vou direto ao assunto: eu faltei com respeito ao irmão do rapaz. Quer
saber o porquê? Ele vende o corpo no centro da cidade.

Certo dia trombei com ele na minha rua e disse: --- é por isso, mano,
que você gosta de se embelezar depois de ficar sarado na malhação?! ---
escuta aqui, mano, esta é a correria que você sempre esteve a fim de
fazer?!

Eu só quis zoar um pouco, e até aí nada de mais, cada um na sua. Mas nós
começamos a discutir. Os que estavam com ele se juntaram no bate"-boca e
disseram que eu estava tirando ele e era folgado. Eu tentei parar a
discussão. Impossível. Além disso os outros me provocaram pra repetir o
que eu pensava mesmo do rapaz, ali, na cara.

Aceitei a provocação, encarei e ofendi a honra do rapaz. Insinuei que
ele devia gostar daquilo --- qual é a tua, mano? --- e disse que ele
devia ter seu lado flor pra oferecer… a bom preço… a sua
beleza. Cada um dá o que tem!

Depois fiquei rindo sozinho. Seguiu um silêncio geral. Os manos da outra
rua me olharam atravessados. Calma aí! Eu saquei o clima pesado. Pedi
desculpa. Não resolveu. Eles tomaram as dores. O~ofendido baixou um
pouco a cabeça, depois me encarou com um ódio da porra. A~faísca foi
acesa.

Quer saber melhor? Se eu estiver armado e o rapaz da outra rua não
acertar o pagamento direitinho e ainda me olhar torto, apago ele do lado
de seu irmãozinho garoto de programa, tá ligado!? O sangue de um espirra
em cima da flor que é o outro.

Tive vontade de fazer isso, mas em outro momento quis deixar quieto.
Ainda mais porque entreguei as armas quando entrei na igreja. Não desejo
virar cristão de arma na mão.

Só que, pra me garantir na negociação, ando junto com alguns moleques lá
da rua, com todo mundo vendo, e alguns manos armados, é claro.

Porque a vila é um mundinho pequeno onde o grande perigo é eles
perceberem que alguém pode virar carta fora do baralho. Principalmente
sendo curinga perdido em outras quebradas.

Daí a hora sinistra do golpe chega mansa quando eles querem tomar o seu
lugar, e a danação vem rapidinho quando um fraqueja, e os outros, na
maldade, detonam já sabendo de tudo antecipado.

Então o forte deixa de ser forte e é devorado. Lei da selva.

Mas vou seguir um conselho que sinto vir do senhor: negociar.

E antecipo agora minha chegada pra fazer o acerto e fechar o negócio.

O senhor pode ver a cena: tudo ao redor continua em seu ritmo; os
moleques brincam soltos na rua ou seguem atrás de uma bola; os
trabalhadores caminham pro trabalho na rotina de sempre; as donas de
casa falam da vida dos outros; os vira"-latas fuçam nos sacos de lixo.

De repente aparecem os caras que ainda estão do meu lado e os caras que
apoiam eles; os dois grupos caminham a partir das duas pontas da rua,
cada um bem do seu lado, até chegarem no meio onde a gente marca um
campinho de jogo com as travezinhas.

Dia normal? Não é. As aparências enganam!

Se o mundo é outro jogo, torça pelo meu time. E~se o jogo não tem juiz
eu quero estar com Deus. Quanto aos outros não sei da companhia deles.

\section{Terceiro tempo}

Eles souberam da treta com o irmão do meu parceiro porque tudo vaza na
vila. Então eles deram um tempo empurrando com a barriga no calor da
discórdia e pra dividir a gente. Tá ligado?!

Foi pior. Adiantaram parte da grana do pagamento da biqueira na pura
esperteza, fazendo pressão pra continuar invadindo nossa área e dominar
território, e ainda puseram uma condição pra parar de invadir: a gente
devolver essa parte do dinheiro.

A gente devolveu a parcela, eles adiantaram uma grana de novo e
repetiram a ameaça.

Eles dão com uma mão e tiram com a outra, trocam uma parte pelo todo e o
todo por uma parte, e mandam avisar que está tudo acertado. Ou mandam
recado dizendo que nós estamos devendo e concordam em deixar limpo desde
que a gente entregue o ponto!

O esquema é dividir e confundir pra conquistar; estratégia de provocação
de guerra que, no fundo, é luta pelo poder, né?

Em resumo: eles dizem que já compraram e a gente diz que ainda não
vendeu. Na maldade e na esperteza eles empurram os acordos pra frente.

Eu continuo firme na minha decisão, e ofereci pro meu parceiro largar
minha parte daquilo que os outros vão ter que pagar. Mas ele não quer
receber a parte dele a não ser que eu também exija a minha.

Então eu disse: escuta aqui mano, eu encaro a perda, abro mão da minha
parte na biqueira pra acertar a pendência daquela antiga partilha com
teu irmão no assalto. Ele bateu pé: não aceita, e ademais já está
armado. Não somente ele, mas os outros também.

Eu percebi: se me querem na luta e eu desejo cair fora estou começando a
desconfiar que já não posso.

Ainda vieram falar em reparação pela ofensa contra o irmão do rapaz.
Deram uma de juiz e decidiram um valor que exigem descontar do preço da
biqueira!

E todo mundo aqui sabe que o bicho está solto devorando dos dois lados,
tá ligado?! E tem mais: se eu nego fogo pode acontecer também de eu
brigar com meu parceiro atual. É~a danação final. Pois o irmão dele está
pra sair da cadeia, continua mandando umas pipas, atiça os
desentendimentos, e quer chamar as facções, principalmente os irmãos do
partido pra apurar tudo.

O senhor sabe que este é um jogo de guerra que faz parte da história do
mundo. Por isso as tretas vivem soltas e de cada uma delas nasce uma
outra. Por isso a briga se multiplica e nunca acaba. Ela só amansa e
depois a gente volta a se pegar.

Pra quem chega a vila parece calma. Não é. Quando o senhor vê a gente
batendo bola na rua parece que é só alegria. Engano. Quando o senhor vê
os moleques fumando baseado na maior cara de pau, parece que é tudo
normal. Outra ilusão feito a pira do bagulho da droga que é a ilusão do
barato. Feito as viagens na noia que a gente faz brisado ou bem louco e
sem sair da quebrada onde a gente se esconde.

Por isso não confio em ninguém, mesmo que eu vá na igreja, tá ligado?!
Mas quando atravesso a porta do templo sagrado caio de novo na real do
mundo, retorno pra minha rua, vejo meus trutas, meus inimigos, os que
podem ser meus amigos, e os falsos amigos, e gente que a gente conhece e
não confia.

A guerra está chegando. E~se um ou outro lado não aceita a negociação
--- dois palitos --- o estrago está feito e segue a detonação, tá
ligado?!

Porque as pessoas querem levar vantagem, dar o golpe na hora que acham
certa e ainda não ficam contentes de chupar teu pescoço. Desejam a
satisfação do confronto feito a adrenalina do crime que tem muitas faces
e está em todos os lugares -- show de horrores.

Daí eu começo a cismar quando escuto uma fala dentro de mim. É~a voz do
cara que eu rasguei de faca e continua enterrado na minha cabeça.

A voz exige um acerto de contas. Daí não sei mais em quem acreditar e
estou virando um forasteiro em minha quebrada. Por isso posso estar com
os dias contados, porque aqui na vila carta fora do baralho deve ser
coberta de terra e quem provou da vida maldita do crime e quer sair fica
com fama de virar cagueta, bandear pro outro lado e até passar pro time
dos gambés.

Então eu fumo pedra durante noites longas de insônia e agonia. Ando em
círculos no meu quarto feito bicho acuado, fera ferida. E~a pira do
bagulho da pedra acende toda a noia na minha cabeça.

Vejo meus perseguidores na minha frente. Espalhados por toda parte.
Posicionados como num jogo. Escuto suas vozes crescendo. E~passos vão se
aproximando.

Vejo uma tela viva insuportável, sessão de terror. Que nem antigamente
como se fosse ontem quando eu tinha doze anos e não conseguia mais ficar
em casa.

Por isso mesmo eu preciso contar o resto.

Aconteceu num dia na vila quando nem era noite e ainda tinha um pouco de
sol.

Eu trombei com os caras da outra rua. A~gente voltou a bater boca por
causa da conversa da flor. Chegou aquele rapaz delicado. Me olhou com
maldade e ódio. Os caras da outra rua me provocaram de novo pra falar o
que eu achava dele. Eu sustentei o que tinha dito. Eles me encararam. Eu
pedi desculpa mais uma vez. Não resolveu.

Cada um, de cano na mão, marchou na minha direção. Eu disse calma aí,
mano. Andei um pouco de costas. Encostei num muro pedindo trégua.
Implorei pra afastarem os canos de mim. Engatilharam. Achei que iam me
apagar. Falaram que estavam só assustando. Eu baixei a guarda, mosquei,
e recebi um tapa na cara.

Todos riram e ainda se olharam triunfantes. Depois recuaram e foram
recolhendo os canos. Deram as costas se achando seguros. Sabiam que arma
eu não tinha desde que fiquei alardeando meu gesto cristão quando entrei
na igreja; riram com deboche e malícia e disseram que minha conversa era
desculpa de cuzão.

Ainda folgaram me jogando aquele tapa na cara. Que seguiu em palavras
que ferem. Pois o tapa correu de boca em boca, espalhou pela vila, e
quando eu via qualquer um tinha certeza de que era disso que a pessoa
falava. Cada olhar atravessado na minha direção era pra insinuar o tapa.

Um dia eles passaram em frente da minha casa dando gargalhada. Fizeram
depois silêncio e atiraram uma flor vermelha na minha porta. Foi aquele
rapaz manso. Eu tenho certeza. E~ele ainda deve ter dado um sorriso de
maldade e triunfo.

\section{Quarto tempo}

Todos marcham pro confronto…

…e eu sou guerreiro…

…mas tem a voz daquele que eu apaguei…

…e também a voz do maligno…

…então sinto vontade de pegar…

…outra vez no meu cano…

…agora eu vejo tudo…

…meu parceiro tem olhar de maldade…

…ele me empurrou pro confronto…

…quando eu desfiz a sociedade…

…ele passou pro lado de lá…

…ele se juntou ao irmão dele…

…e também se juntou ao rapaz da outra rua…

…que vai assumir o controle…

…com aquele seu irmão manso e delicado…

…que olha fundo nos meus olhos…

…invasor da minha mente e do meu corpo…

…e desfila oferecido pela cidade…

…pra chegar no melhor preço…

…vai abrindo os braços…

…com um jeito de flor…

…anda com sorriso atravessado…

…fazendo o jogo da maldade…

…dá uma balançada no corpo…

…a cartada final é dele…

…porque eu sei de tudo…

…a voz do cara que eu apaguei me revela…

…o rapaz delicado é a isca…

…pra ser completada a danação…

…o sinal de detonar é dado por ele…

…sua língua, feito cobra, dança bailarina…

…uma flor vermelha aparece em sua boca…

…ele cospe a flor no meu rosto…

…eu revido dando a outra face…

…ele apanha de novo a flor…

…ele me bate a flor na cara…

…com a outra mão me acaricia…

…derrama sangue em minha cabeça…

…eu sinto o primeiro tapa na cara…

…eu sinto o primeiro açoite nas costas…

…eu caio no chão e me ergo…

…caio de novo…

…sou erguido…

…sou pregado numa cruz…

…sou o ladrão…

…também o outro ladrão…

…sou o soldado que bate…

…sou o soldado que arranca o sangue…

…com a ponta da lança…

…o rapaz tapa a minha face…

…ele tem a mão aberta…

…ele vai fechando a mão…

…ele me aperta a garganta…

…os outros se aproximam rindo…

…escuto suas vozes…

…vão marchando…

…parecem fantasmas dançando…

…enquanto o sol vai se pondo…

…todos estão gargalhando…

…todos colam no meu corpo…

…você meu irmão me ajude…

…você alguém me ajude…

…você ninguém me ajude…

…você legião me ajude…

…o Senhor me ouça e me ajude…

\emph{…}suplico todos os perdões…

…de joelhos confesso…

…sou pó e nada mais que pó…

…sou pedra e nada mais que pedra…

…usei a mim mesmo…

…fumei a mim mesmo…

…todos me consomem…

…sou eu contra todos…

…mas dentro de mim tem um guerreiro…

…ele é de fé e de luta…

…dentro de mim quer explodir uma vontade…

…do justiceiro que eu sempre quis ser…

…desde quando eu tinha doze anos…

…e via uma cena em minha casa…

…queria parar a cena…

…mas segui tombando pelo mundo…

…sem ter outra saída…

…os tempos mudaram, meu irmão…

…os tempos acabaram, meu irmão…

…os tempos, cada um dos tempos…

…e eu estou aqui na minha toca…

\emph{…}mas sou apenas sombra…

…sou aquele que não é…

…sempre fui o invisível…

..\emph{.}e se eu aceitei a palavra sagrada…

…tenho um medo profundo…

…mas a minha alma tem sede de Ti…

…e ainda preciso dar uns tiros, meu irmão…

…só me resta dar uns tiros, meu irmão…

…só me resta dar uns tiros, meu irmão…

…dar uns tiros…

…dar uns tiros…

…dar uns
t…

\begin{center}\asterisc{}\end{center}
\begingroup\small

\emph{Tudo começou aparentemente na rotina de um atendimento a um rapaz
usuário de cocaína e de crack e metido com a bandidagem e com o
tráfico.}~

\emph{Da mesma forma, aliás, como tenho visto tantas vezes --- seja um
caso de pequeno tráfico oportunista, de tráfico para abastecer o consumo
adictivo ou, então, de ousadias maiores quando existe na berlinda um
sujeito que deseja crescer na criminalidade.}~

\emph{Confesso que este é um assunto complexo, uma vez que os limites
entre esses grupos não costumam ser claros, e também porque, nas áreas
mais carentes da periferia das grandes cidades, o emprego de
adolescentes ou até de crianças no tráfico nem sempre ganha a aura de
crime supostamente hediondo, embora se torne infelizmente opção de
trabalho, não raro com apoio fingido ou até declarado de certas
famílias.}~

\emph{Ademais, há que se reconhecer uma verdade: famílias ditas
desestruturadas pela miséria, pelo alcoolismo ou por outros ``vícios''
não constituem propriamente exceções em meio à acreditada virtude geral
das famílias.}~

\emph{Muito pelo contrário: alguma disfuncionalidade costuma ser a regra
em quase todas as famílias. No entanto, raramente as mazelas domésticas
em redutos periféricos extrapolam para a dramatização escandalosa de
domínio público, o que é também consolação para os que acreditam
(falsamente) que pecam menos, ou é consolação para os mais discretos, ou
então, é consolação para os mais hipócritas.}~

\emph{A~droga genérica do prazer --- e a droga ilícita principalmente
--- entra no cadinho efervescente e econômico por detrás da santidade
suspeita dos redutos domésticos, com a conivência não rara de igrejas e
com a presença do crime organizado que, de costume, congrega pessoas
bastante moralistas que não enxergam incompatibilidade entre
contravenção fora de casa e virtudes conservadoras intramuros.
Exatamente como ocorre nas famílias mafiosas mundo afora. Mas que também
ocorre nas famílias ditas honestas e virtuosas!}

\emph{Então volto ao assunto deste rapaz, que tinha uma capacidade
intensa de evocar suas memórias sem fazer dramatizações excessivas,
seguindo seu caminho de forma firme, contundente, perigosa, ou mesmo
trágica.}~

\emph{O~discurso dele seguiu bem além da mera confissão banal do adicto,
qual seja, o de mencionar as drogas de preferência e suas escaladas e
coisa e tal. Ele chegou ao nível de um cinismo confessional, embora
simplório e periférico, mas permitindo que um atendimento médico pudesse
aos poucos extrapolar para níveis pessoais mais ousados.}~

\emph{Até certo ponto meus encontros com ele se deram conforme o que se
espera do meu ofício; ou seja, um ofício de quem escuta dramas de gente
sofredora. Mas com adictos complexos há que se buscar caminhos diversos
e decifrações, há que se aprofundar a empatia.}~

\emph{Então, de repente, lá surgiu uma proposta muito especial. Era um
pedido que parecia ser normal, porém era um pedido espantoso. Foi quando
o rapaz sentiu ter ganho confiança em mim. Então ele me trouxe um
imbróglio delicadíssimo.}~

\emph{Lá no fundo tratava"-se apenas de uma questão de negócios. Mas era
mais uma tentativa aparente de sair do crime e também uma proposta de
negociação de uma parceria com uma biqueira (ponto de venda de droga).}~

\emph{Eu me senti um pouco coagido a princípio, é claro.}~

\emph{Mas depois o rapaz se despejou em revelações bombásticas e se fez
humilde e ousado, até que eu acabei aceitando uma função temporária e
meio descabida de me transformar num conselheiro à distância. Vejam só
que coisa!}

\emph{Eu realmente acredito que o começo do meu texto seja um espelho
mais ou menos realista da vida do rapaz, com mínimos lances míticos
distorcidos de herói das sombras, e juntando novidades suspeitas quando
certos membros do crime} freelancer \emph{ou semi organizado resolvem abrir o
jogo e advogam em causa própria.}~

\emph{Digo tudo isso sem deixar de lado a questão nebulosa da culpa em
uma ou outra ação criminosa chocante, ou ao menos chocante para quem não
está acostumado a conviver com pessoas nos limites da miséria ou pessoas
angustiadas nos limites da explosão por terem o pavio curto.}~

\emph{Pois assim, em meio a confissões contundentes de franca maldade e
também de arrependimento, e entre os apelos fortes do rapaz, eu fui
cedendo aos pedidos de uma consultoria especial.}~

\emph{Não demorou e eu já estava caminhando no fio da navalha,
procurando não me comprometer. Buscando soluções enquanto apelava ao
velho bom senso, à sabedoria perene, à própria autoajuda.}~

\emph{Admito que aquele caso tinha uma virtude especial de ilustrar não
apenas uma busca de redenção, mas era um caso que refletia um
microcosmos de interesses escusos em plena Zona Leste de São Paulo. Digo
microcosmos (e micronegócio) conforme a velha luta econômica por objetos
do desejo, conforme a velha luta pelo poder e por bens e mercadorias
vulgares. Sem dizer que esse microcosmos seria, enfim, uma réplica de
outro macrocosmos sempre presente na história política e econômica do
mundo.}~

\emph{O~}link \emph{com a História teria sido meu estímulo e meu ponto de
partida, e assim fui me dando conta de um atoleiro de problemas
interdependentes dentro dos limites realistas do que se tem como ética
do crime e do que se tem como oportunismo na quebra dessa ética.}~

\emph{Ambos os casos --- a ética e a sua quebra --- na verdade existem
simultaneamente. Isso é até meio dialético. Pois toda ética pressupõe o
convívio com a ausência dela mesma. Ainda mais quando é fato que o crime
não é tão organizado quanto parece ser.}~

\emph{O~crime tem uma dinâmica de rupturas e estrangulamentos, de
loucuras e ousadias, principalmente quando os seus protagonistas são
tipos que nem este rapaz --- um quase zé"-ninguém meio esquecido e mais
inclinado a ser um} freelancer \emph{da contravenção, e tão adicto de
crack/cocaína quanto adicto de novidades e ainda querendo curar"-se das
mazelas de família e dos seus complexos de criação.}~

\emph{Tudo isso não é novidade alguma. E~ainda é assunto oportuno que
não consta em noticiários. Assunto esquecido que escapa às matérias
estereotipadas denunciadas na imprensa.}

\emph{Enfim, eu fui compreendendo aos poucos o nível de complexidade da
trama e do drama, em que as desculpas de tipos ditos marginais acusam
dilemas ``cabulosos'', até mesmo dilemas algo dostoievskianos nas
``quebradas'' da Zona Leste, com apoio da linguagem regional, com
menções sucessivas de ``mano'', ``trairagem'', ``crocodilagem'', com
repetições frequentes do ``bagulho loko'' e variações.}~

\emph{Daí eu bem percebi que as visitas do rapaz iam se encaixando numa
lógica operacional popular. E~foi assim que eu me via na imposição
crescente de não recusar o papel de conselheiro ou de ``consultor
empresarial'', e sendo minha pessoa investida pelo rapaz de uma escusa
paternidade passageira. Ah, sim, talvez seja esta a questão principal.}~

\emph{Psicanaliticamente falando, havia ali também uma transferência, e
por que não?!}

\emph{De uma maneira ou de outra, entre uns e outros papéis assumidos
por mim perante ele, eu procurava tanto a fina argúcia como a oportuna
frieza, como também procurava, sei lá, uma grandeza de alma para
transcender a mesquinharia da coisa sórdida do tráfico. Que nem era
coisa tão mesquinha, ou nem era tão sórdida, mas era terrível e
implacável.}~

\emph{O~rapaz voltou algumas vezes, e cada vez trazia um temor à flor da
pele como se a próxima etapa fosse uma ausência esperada. Como se cada
etapa fosse uma execução anunciada.}~

\emph{Não estou exagerando nem um pouco, caro leitor. Eu deixei de lado
a função médica de tratar do rapaz como um simples dependente de drogas
porque ele nem era mais isso, ou se era, essa condição já não parecia
relevante frente às armadilhas do destino construídas para ele com sua
cumplicidade.}~

\emph{Foi quando decidi mesmo centrar o tratamento naquela consultoria
excêntrica esperando, até de maneira caridosa e cristã, conduzir uma
ovelha de volta ao rebanho ou solidarizar"-me com algum} mea culpa \emph{do meu
interlocutor e torcendo para que suas supostas bravatas de
arrependimento dessem bons frutos nas searas do perdão.}

\emph{Sei lá se deram bons frutos ou maus frutos. Nem sei mesmo no que
deu o imbróglio todo, e temo um final que para sempre ficará sendo uma
incógnita. Um final a me cutucar insinuações de tragédia brasileira para
que eu possa fazer apenas alguma literatura como consolo.}~

\emph{Ainda pensei que tudo aquilo virasse um monólogo. Cheguei a
imaginar um cenário de uma toca, e alguém com gorro ninja frente a
pacotes de drogas ao lado de uma Bíblia.}~

\emph{Mas com ou sem teatro faltava um elo incógnito: era a questão
daquele rapaz que era irmão de um outro que queria comprar a biqueira.
Com esse outro rapaz teria havido um desentendimento terrível e cabuloso
que o meu paciente omitiu e eu nem perguntei detalhes porque havia uma
reserva no discurso dele.}~

\emph{Então precisei criar um pouco de ficção, mas nada comprometedora e
se é que é ficção mesmo. Criei a história do garoto de programa e ali
encontrei um símbolo apropriado, a flor vermelha, a flor cor de sangue,
o pomo da discórdia numa história muita tensa entre machinhos.}~

\emph{Sei que houve ali um mistério e um fator da danação e um núcleo
explosivo da maldade. Sim, porque realisticamente tudo caminhava para o
confronto, para o ajuste de contas, para a guerra.}~

\emph{E já vislumbrando a guerra eu me vi convidado a ficcionalizar o
final bélico escolhendo caminhos através dos delírios e tormentos
propiciados pela droga pesada, pelo crack. Sem talvez fugir muito ao
personagem real que juntava o sonho à necessidade, que juntava a
fantasia e a loucura à realidade.}~

\emph{Enfim, não há mais tanto a dizer, a não ser que ainda me vem à
mente a figura dele --- jovem, magro, esguio, misterioso -- uns vinte e
seis anos mais ou menos; cheio de orgulho típico de gente crescida no
crime e na exclusão, de gente que reage com tentativa de grandeza para
encobrir a própria pequenez e a fragilidade.}~

\emph{Não posso deixar de assinalar uma angústia que ainda me toma
profundamente quando revejo meu papel de ter sido um conselheiro
atípico. Chego a relevar considerações a respeito de transtornos, de
desvios de caráter e coisa e tal. Pois fica muito difícil separar o joio
do trigo e a virtude do vício.}~

\emph{Sim, porque tudo isso se embaralha no fundo complexo das
vicissitudes sociais, e de tal maneira que as avaliações, ou mesmo os
diagnósticos, mais uma vez devem ser revistos com cuidado.}~

\emph{Ainda mais em um mundo marginal complexo com pitadas
dostoievskianas misturadas com pitadas de tragédia grega, entre culpas e
castigos judaico"-cristãos frementes no meio da lei do cão e à margem
suspeita das igrejas evangélicas, onde tantos lavam rotineiramente suas
responsabilidades da tal ``vida errada'' e, além disso, procuram
qualquer salve"-se quem puder ou salve geral quando o bicho pega solto.}~

\emph{Eu concluiria dizendo que este nem é um texto especificamente
sobre o dependente de crack, ou não seria talvez uma história comum sobre
os descaminhos do crime porque é história demais comprometida com o mito
do herói novelesco com a sorte invertida.}~

\emph{Mas para mim foi e continua sendo um momento especial de uma
reprodução capenga e um tanto estilizada do mundo maior, do mundo
político belicoso, da História Universal, dos jogos maquiavélicos de
poder no calor da busca por todos os objetos do desejo, por todos os
tipos de mercadorias disponíveis no planeta.}~

\emph{Um mundo onde a droga substância mais uma vez se funde com o resto
das coisas que podem ser droga, e que podem ser as mais genéricas drogas
possíveis. Como na história da lágrima atrás do gorro ninja e na história
daquele rapaz que viu a execução brutal de seu amigo.}~

\emph{Creio que por tudo isso tenha sido melhor dar um fecho com alusões
bíblicas mesclando uma busca indiscriminada do crack a outros caminhos
existenciais suspeitos.}~

\emph{No entanto, nada disso me apaga a tensão por um final que jamais
me será revelado.~}
\endgroup