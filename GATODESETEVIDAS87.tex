\chapterspecial{Gato de sete vidas}{}{}
 

Disse meu amigo e escritor Wilson Luques Costa, numa conversa de bar, no
Jardim Santo Estevão, Zona Leste de São Paulo, que ``há momentos na vida
em que a única coisa que não parece real é a própria realidade''.

Eu acrescentei que, entre um fato recontado e uma ficção possível, estão
inclusos imprevisíveis desdobramentos do que se chama por aí de
``Destino'', sendo ``destino'' uma palavra vaga para tudo de importante
ou para nada de especial que ocorre no mundo, e de tal maneira que esse
assunto permanece velho e redundante quanto o próprio mundo.

Meu amigo ainda lembrou que os gregos antigos personificavam o Destino
nas Moiras, e Moira, no singular, é o quinhão reservado a cada um em sua
trajetória desde o nascimento até à morte. Isso não está longe do que os
hindus chamavam de \emph{Karman} ou Roda do Samsara, por onde se faria a
presença de \emph{Maya} criando formas transitórias dentro de um imenso
palco em que os pobres mortais --- acreditando"-se donos de suas ações
--- seriam marionetes dos deuses ou vítimas de si mesmos.

\asterisc{}

Desejo contar agora uma estória simples que começou durante a procura de
um ser humano desgarrado --- um jovem dependente de drogas, amigo meu,
que vivia num estado de semi"-abandono e deixou de fazer parte de meu
grupo de teatro. Eu, por razão mais pretensamente cristã do que helênica
ou hindu, me vi na contingência de resgatá"-lo.

Ao saber que ele freqüentava um bar aqui no bairro, fui até lá e não o
encontrei. Mas seus rastros se transformaram num atalho para outra
estória em que sou a testemunha principal, se não a única. Porque sou
aquele que viu, ouviu, esteve presente em lugares diversos e fez certas
ligações no tempo e no espaço.

\asterisc{}

Quando cheguei no bar um tipo atarracado e baixote colou em mim. Ele
derramava um copo de cachaça goela abaixo, orgulhoso de sua performance
etílica, e buscava plateia sem ter consciência de estar sendo um pobre
diabo fazendo um patético teatro naturalista e banal.

Ele era bastante exibido. Extravasava emoção torta em grotescas
gargalhadas. Tinha os dentes quase todos faltantes e aproximava seu
rosto do meu de maneira inconveniente, ao soltar bafo de álcool.

Parecia um Sancho Pança com sotaque nordestino, dando show numa mísera
taberna, roubando cena e mal imitando a eloquência de Don Quixote. Mas
seu discurso não tinha grandeza ou nobreza. Tinha uma horrenda clareza.
E~ele defendia uma proposta corretiva cínica e cruel como ``solução
perfeita'' para certos desregramentos.

Porque hoje em dia --- bafejava --- não se podia criar menina moça sem o
risco dela virar pasto de macho garanhão! Mas, para combater tamanho
abuso, nem era preciso pena de morte, fosse por linchamento, tiro, ou
lâmina rasgando as tripas do maldito. Era suficiente a radical castração
corretiva de se arrancar, ao pretenso Don Juan, sua folgada macheza.

O homem ainda berrava: ``não estou certo? não estou certo?'' --- dando
resposta afirmativa para si mesmo. Tremia todo e, entre um gole e outro
de cachaça, vociferava, com grotesco sorriso, a frase ``nem precisa
matar''. E~gargalhava e babava, detalhando o exato momento em que seriam
extirpados o órgão masculino do vagabundo mais os colhões. Daí estendia
uma mão em garra para fazer porca mímica da retirada.

Pois ele deixaria o safado liso, ah, ele o deixaria reduzido à condição
artificial de ter virado uma ``menina''. ``Ah, nem precisa matar'' --- e
repetia a frase fazendo uma careta estupidamente dionisíaca antes de
pedir outro copo de cachaça.

Finalmente, ele anunciou que nada restaria ao vagabundo a não ser um
buraco entre as pernas. E~concluiu com uma expressão sumária utilizando
o mais curto e popular palavrão em língua portuguesa --- mandando uma
mensagem bar afora de como seria a disponibilidade sexual de uma
criatura transformada num restolho.

Eu me afastei, assustado. De longe, o homem continuou a me fazer de
plateia. Perguntava continuamente para si próprio se estava certo.
Respondia imediatamente que sim, enquanto gesticulava bastante com um
novo copo de cachaça na mão.

\asterisc{}

Dias depois o homem cruzou de novo meu caminho. Minha surpresa foi vê"-lo
do outro lado da minha mesa, sóbrio, vivaz e simpático, um pouco sem
graça quando admitiu ser um dependente de álcool. Não me reconheceu e eu
não revelei minha presença no bar.

Antônio ele se chamava. Serralheiro, do sertão da Paraíba, residente há
muitos anos em São Paulo, mal aposentado. Vivia uma penúria de vida que
só vendo, sem dizer que em sua família havia um drama cujos começo e fim
são iguais aos de muitos outros.

Antônio tem uma filha. A~filha de Antônio conheceu um rapaz. O~rapaz
tinha muito charme e era usuário de drogas, principalmente de crack. A~filha de Antônio cedeu à lábia do rapaz. Houve encontros em embalos
festivos. A~barriga da menina cresceu, e ela pariu um menino filho de
mãe solteira.

Quanto ao resto do que sucedeu, de alguma maneira escusa deu"-se o
esperado: o rapaz agiu como muitos aventureiros de caráter duvidoso que,
com o consentimento de uma menina nada inocente, aproveitam"-se da farra
e caem fora no mundo.

Antônio é um homem simples, emotivo e conservador. Sentiu ímpetos
homicidas vindos do seu gênio sanguíneo de paraibano da gema. Engendrou
planos de eliminar o rapaz como se brotasse, no seu íntimo, um vingativo
código de honra de seus ancestrais lá dos sertões. No entanto, ao chegar
quase às vias de fato, hesitava no vai e vem do calor e do resfriamento
da vontade.

Naquela ocasião Antônio começou a frequentar uma igreja evangélica.
Virou ``crente'' e logo passou a fazer confidências ao pastor como se
fosse de igual para igual. Resolveu confessar seu desejo de fazer
justiça com as próprias mãos. Expôs seus planos. O~pastor ouviu
atentamente. Ao final da conversa Antônio foi contestado com interdições
bíblicas e refreou seus impulsos.

O rapaz, porém, não lhe saía da cabeça. Mesmo porque ele surgiu de novo
em seu círculo familiar --- gato das ``quebradas'' --- nesses ermos do
Jardim Santo Estevão, e tornava a ver a filha de Antônio feito raposa
num galinheiro, e a menina --- volúvel e encantada --- o acolhia.

Antônio admitiu que foi ficando sem ação com o desenrolar dos
acontecimentos. Começou a sentir"-se um otário. Mal conseguia conter seu
ódio. Conversava bastante com o pastor. Então ficava um pouco calmo, mas
o rapaz aparecia de novo, maneiro e insinuante. A~filha de Antônio não
resistia. Mesma velha estória.

Num belo dia Antônio surpreendeu a si próprio: cansou"-se do ódio e
resolveu dar uma chance a estes Romeu e Julieta, que seriam apenas de
longe parecidos com as personagens de Shakespeare.

Se Antônio não tinha cultura erudita, já ouvira falar de Romeu e Julieta
e de outras estórias que viraram contos populares universais. Sendo um
homem crédulo e um homem conservador e apreciador de cerimônias, não
resistia em querer fazer da vida uma romântica ficção. Tanto foi assim
que, por súbito capricho da vontade, propôs a celebração de núpcias
tradicionais: --- que os dois se casem e o pai assuma o filho!

Foi marcada a data do casório e Antônio fez os preparativos com
paciência e empenho. O~Diabo, no entanto, interferiu no assunto
familiar, conforme insinuou o pastor, que tinha um gosto especial de
acompanhar o rumo dos acontecimentos.

Logo antes da cerimônia de casamento o rapaz caiu fora e seguiu seu
caminho rumo a outra gandaia. E~se ele deixou a filha de Antônio desesperada e revoltada, deixou"-a
também com outra criança na barriga, para se amigar em seguida com uma
mulher bem mais velha --- uma ``coroa'' aparentemente separada de um
traficante que, apesar de um divórcio incompleto, tinha ainda essa
mulher como eventual amante.

Nesse suspeitíssimo triângulo o rapaz deu"-se bem no início. Empolgado
pela aventura e ingenuamente amalandrado na experiência passageira de
gozar benesses proibidas, ele passou a trabalhar com sucesso em tarefas
consideradas malditas. Aproveitou"-se da receptividade quente da
``coroa'', que suspirava de gozo lhe abrindo não somente as pernas como
também lhe abrindo portas comerciais em becos e vielas.

O rapaz começou a cuidar de uma rede de ``biqueiras'', ou seja, pontos
de venda de droga. Virou gerente de tráfico, e subiu na vida favorecido
por uma posição oportunista num triângulo amoroso em que ele não somente
fazia um papel secundário e subalterno de traficante, como também de
cafetão de araque.

O rapaz tornou"-se muito vaidoso. Andava com tênis de grife, colares,
pulseiras, óculos escuros de aro prateado. Intimidava a vizinhança que o
temia e respeitava, porque ele carregava uma \emph{\versal{PT} --- 9 milímetros},
e gozava um status prematuro de bandido em fulminante carreira.

Acontece que ele se empolgou tanto que caiu numa armadilha. Conforme
diria o Zé Povinho, ele não viu a sombra do negócio e topou com a mulher
do cão! Mas ele deveria saber disso, e aceitou as regras do jogo como se
mal soubesse do preço a pagar.

Veio então a danação e tudo se deu como tudo teria que se dar.

O ex"-marido da ``coroa'', que já tinha sugado o que conseguira de quem
Antônio chamou com menosprezo de ``laranja'', resolveu dar um basta no
que tinha sido tanto uma aventura amorosa fugaz quanto uma carreira
profissional falsamente promissora.

O traficante aguardou o rapaz numa viela escura quando este passou
montado numa moto novinha. Vários tiros reduziram uma farra de três a um
jogo de interesse de dois.

O traficante contemplou o corpo imóvel que sangrava por vários orifícios
e apanhou de volta a \emph{\versal{PT} -- 9 milímetros}~enterrada na cintura do
rapaz.

Mas ele não apenas resistiu aos tiros, como apareceu vivalma disposta a
ajudar. Levado ao hospital, agonizou desenganado.

De resto, ainda vieram desse caso umas surpresas, como também é bem
verdade que certas notícias correm rápido nesses redutos periféricos.

Tão logo soube do acontecido, Antônio passou por uma escalada de
sentimentos.

No primeiro dia sentiu um descarrego de fúria contida.

No segundo dia culpou o rapaz pela tragédia.

No terceiro dia lembrou"-se dos preparativos nupciais.

No quarto dia interessou"-se pela agonia de um sobrevivente.

No quinto dia pensou em ir à igreja.

No sexto dia foi à igreja e conversou com o pastor, que aconselhou uma
reparação.

No sétimo dia foi incumbido de uma missão religiosa.

\asterisc{}

Antônio anunciou"-se no hospital como parente, pegou um crachá, seguiu
corredor afora, e entrou numa enfermaria coletiva. O~rapaz estava num
canto, cercado por um biombo. Não havia ninguém a seu lado, e até então
nunca aparecera visita.

Antônio aproximou"-se cerimoniosamente do leito, removeu o biombo e
sentou"-se numa cadeira ao lado da cabeceira da cama. Manteve uns minutos
de silêncio, recapitulou conversas com o pastor, abriu a bíblia e citou
algumas passagens.

Permaneceu um bom tempo na enfermaria numa atitude piedosa de vigília.
Os outros visitantes olhavam em silêncio, e os pacientes também,
erguendo"-se de seus leitos, curiosos e compadecidos.

Como Antônio vem de uma cultura que gosta de esbanjar a palavra, ele fez
um pequeno discurso. Falou alto, pausado, mas foi breve: ``bem, meu
filho, você vai ter que ir embora deste mundo porque é escolha de Deus.
Você vai morrer, meu filho, e precisa aceitar a morte. Mas tudo o que é
seu vai ser meu. De hoje em diante eu substituo você como pai''.

O rapaz não se mexia e mal respirava. Antônio colocou uma mão espalmada
na testa do doente e deslizou algumas vezes a mão para cima e para
baixo, carinhosamente.

Os olhos de Antônio brilharam úmidos. Duas lágrimas pesadas, viscosas,
difíceis e hesitantes verteram de uma outra face imóvel, lágrimas que
Antônio fez questão de não enxugar, deixando que rolassem aos
pouquinhos.

Quando Antônio se retirou da enfermaria, as pessoas aglomeradas na porta
abriram alas com respeito. Ele seguiu calmo pelo corredor, marchando
firme, indiferente aos que transitavam afobados. Largou o crachá na
recepção do hospital, e seguiu pelas ruas impassível, sereno, decidido.

Na virada do sétimo dia o rapaz morreu.

\asterisc{}

Eu olhei no relógio e verifiquei que Antônio estava há um bom tempo na
minha sala. De repente ele se levantou e pediu desculpas, não pela demora,
e sim por ter de ir embora. É~que ele tinha uma obrigação. Era quase
hora da filha fazer um bico de serviços domésticos numa casa de família,
e ele precisava ficar com seus netinhos. Ia dar umas voltas na praça com
uma menina de colo de três meses, muito bonitinha, e com um menino de um
ano e meio também muito bonitinho.

Antônio fez questão de comentar que, na sua idade, era difícil conseguir
emprego porque davam valor apenas aos jovens. Mas por isso sobrava tempo
para ele cuidar de crianças, mesmo que fosse para dar passeios na praça
como simples avô.

Antes de se despedir ele fez uma observação breve sobre a dor inevitável
do ato de viver, e até recapitulou, com espontaneidade nordestina, um
ditado oriental que diz assim: ``viva com alegria em meio às tristezas
do mundo''.

\asterisc{}

Passaram"-se alguns dias e eu soube que aquele meu amigo desgarrado do
grupo de teatro foi achado desfigurado numa viela, num local bem perto
de onde tombara o primeiro rapaz.

O meu amigo, que tinha idade próxima à do outro, foi brutalmente assassinado,
e em circunstâncias desconhecidas até o momento, aparentemente sem testemunhas do crime, e sem que alguém o tivesse ao menos socorrido antes do final da sua agonia, que foi o mais breve que os sete dias do outro, e teria durado umas sete horas, desde aproximadamente meia noite até pouco depois do amanhecer.

\asterisc{}

``Sim'' --- repetiu meu amigo escritor naquele encontro de bar, aqui no
Jardim Santo Estevão, Zona Leste de São Paulo: ``há momentos na
vida…''

Acabamos tendo uma longa conversa, se não uma discussão filosófica, sobre
o que é realidade, e voltamos a falar de gregos e hindus.

Ao sair do bar, eu vi Antônio entrar. Ele não teria percebido que nossos
caminhos se cruzavam, mais uma vez, de passagem!

\begin{center}\asterisc{}\end{center}


\emph{Estória, praticamente toda baseada em acontecimentos reais, é a
única em que não vi e não conheci um certo tipo usuário ou dependente de
crack.}~

\emph{No entanto, eu me senti tomando conhecimento dos sucessivos
rastros insidiosos desta pessoa, ou de sua suspeita passagem através de
um mundo cão, enquanto me fixava no relato do serralheiro Antônio (nome
e profissão fictícios), e enquanto pude me valer de um suposto acaso a
me colocar esse homem à minha frente, todo bebaço e grotesco no bar e,
depois, todo sóbrio e educado, no consultório.}~

\emph{Quanto àquele meu amigo, ele esteve no meu teatro e não sei por
que motivo sumiu. Minha busca por ele começou exatamente daquele jeito
descrito. E é verdade que ele foi barbaramente assassinado. O~real
motivo não se soube.}~

\emph{Mas não pense o leitor existir necessariamente uma relação direta
entre a morte dele e a do outro rapaz. Ou, então, imagina"-se existir uma
relação.}~

\emph{O~que fazia parte do meu teatro era filho de um chefão do crime
respeitado no bairro. Mas sua estória de vida perturbada, no meio do
abuso de drogas, não chegava a endossar o dito popular do ``tal pai tal
filho''.}~

\emph{O~filho, sem ser bandido, mas sendo um sofisticado malandro,
valia"-se apenas da notória paternidade para ganhar um respeito local. De
resto, o que eu sei é que ele deve ter vivido uma vida ``loka'' aos
trancos e barrancos.}~

\emph{Eu nunca pude saber se a ascendência suspeita desse meu amigo teve
algo a ver com seu assassinato. No entanto, como naquele lugar acontecem
coisas que até Deus duvida, é lá que uma incerteza cruel dá as mãos a um
mistério tenebroso.}~

\emph{Porém, mudando o foco da questão de alhos para bugalhos, posso
seguramente afirmar o seguinte: se eu me coloco na mixórdia desta
estória como médico, e caso eu me obrigue a opinar sobre diagnósticos,
pouco teria a dizer sobre o rapaz que foi quase genro de Antônio.}~

\emph{Estou certo de que ele deve ser um tipo a me lembrar vários outros
adictos bastante problemáticos, também rotulados muitas vezes de
``sociopatas'' em razão de seus procedimentos anti"-éticos e
desrespeitosos com relação aos valores morais vigentes.}~

\emph{Mas sabe"-se lá se este tipo liso e malandro é ou não um sociopata;
sabe"-se lá se ele tem uma marca distinta perante outros tantos
aventureiros pobres diabos da vida.}~

\emph{Eu não tenho elementos seguros para dizê"-lo, e prefiro antes
admitir que ele me parece mais um desses emergentes das misérias e
agonias familiares; ele parece ser mais um dentre tantos adictos
``buscadores de novidade'' que é faltoso de caráter, voluptuoso e com
traços dionisíacos.}~

\emph{Ou ele seria mais um com jeito de Hermes rapina a transitar no
universo dos prazeres de alcance imediato. Porque ele, como personagem
obscuro de si mesmo, é parecido a tantos que conheci bem e são
estimulados pela cultura brasileira do levar vantagem, são atraídos pela
ilusão fácil do crime, e costumam por vezes ser também indiferentes ao
risco de morrer, além de serem candidatos à tragédia -- tragédia essa
que eles podem antecipar com mórbido prazer. Porque isso faz parte de um
estranho jogo.}

\emph{E se esse assunto parece uma porrada na alma, trata"-se de uma
realidade bastante visível para quem lida com adictos, ainda mais nessa
periferia distante de Sampa que pode fabricar, ou mesmo parir, supostos
``sociopatas''.}~

\emph{Mas o assunto é confuso, senão complexo. O~rótulo de ``sociopata''
por aqui é vago sempre que há uma tentativa de apropriá"-lo do ponto de
vista médico e transformá"-lo em diagnóstico. E, no caso do rapaz, há um
motivo razoável: sua ousadia é sintônica com certos ambientes
socialmente aceitos; é uma ousadia que pode ser bem acolhida perante os
códigos invejados e comuns do machismo vulgar; é uma transgressão
habitualmente reforçada por uma cultura popular. E~apesar das
interdições morais e religiosas, essa cultura faz eco ao desejo legítimo
de muitos aventureiros bem colocados socialmente que se consideram
bastante moralistas e religiosos.}~

\emph{Na verdade, a questão segue um pouco além. Eu imagino esse
relativo desafeto de Antônio chegando a posar, perante esse mesmo
Antônio, como sendo seu alter ego radical, na medida em que Antônio é
também um ``cabra'' nordestino que virou trabalhador em Sampa, tomou
para si zelar por uma moral familiar, comum e conservadora, mas tem
nostalgia e alguma inveja daquele a quem quer confiscar tudo na hora da
morte.}~

\emph{É forçoso reconhecer que, lá no fundo, o conservador e o
transgressor trocam conteúdos e são mais interligados do que se
imagina.}

\emph{Para mim, nada mais significativo para expressar a natureza desta
estória do que a antítese complementar ódio/amor perante a qual Antônio
deu, através de seu discurso, uma cínica e literária visibilidade ao meu
texto, entrando num domínio dos seus desejos sombrios dentro dos quais o
fervor religioso de um homem junta"-se ao desejo homicida dos que se
valem das vendettas familiares conforme códigos ancestrais.}~

\emph{Na minha opinião tudo isso é um pouco do espírito dos sertões do
Nordeste presente em Sampa, presente na periferia Leste, e distribuindo
papéis entre supostos bandidos ou malandros por um lado, e trabalhadores
tidos como certos e honestos por outro. No entanto, esses grupos
podem também se confundir ou trocar papéis.}

\emph{Existem tantos Antônios como existem gatos de sete vidas, numa
complementaridade suave/infernal. Com uns rondando os outros no meio de
muitas filhas volúveis virando iscas para o bote de garanhões
malandros.}~

\emph{Eu cinicamente confesso que, quando Antônio me descreveu a cena do
hospital e me disse aquelas palavras (citadas no texto com a exatidão
com que foram ditas) --- de modo a reunir piedade e crueldade
simultâneas --- bem, caro leitor, aquilo foi o estopim para a crônica.}~

\emph{Mas no meio de tantas estórias supostamente reais, é raro que a
tão hindu Roda do Samsara e que as tão gregas Moiras (conforme a minha
conversa inicial) tragam facilmente à tona um enredo grotescamente
romântico e peculiarmente vulgar, e com um ápice dramático de crueldade
piedosa tal como se deu no hospital na véspera da morte do rapaz.}~

\emph{Caso eu novamente encontre aquele meu amigo escritor em algum
boteco, poderemos conversar mais sobre esta estória e sobre o Destino.}~

\emph{Então, já antecipo uma conversa na qual uma dúvida ligará os
fios do improvável que ainda é parte do possível. Digo do possível torto
e incerto e do que não deixaria de ser, ao mesmo tempo, meio ficção e meio
realidade, e como não!~}
