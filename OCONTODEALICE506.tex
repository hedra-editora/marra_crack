\chapterspecial{O conto de ~Alice}{}{}
 

Alice é tímida, simplória, e mal sabe ler e escrever. Seu pai, irmão e
tio são alcoolistas, sua irmã foi internada com depressão, e uma filha
tentou se matar aos treze anos.

Até vinte e sete anos Alice não fazia uso habitual de qualquer droga.
Mas já tinha passado por sérios problemas, feito uma primeira tentativa
de suicídio aos treze e uma segunda tentativa aos dezesseis com ingestão
de fenobarbital.

Quando criança, Alice conheceu um menino que morava na mesma rua. Os
dois tiveram namoricos que foram proibidos por pressões familiares.
Devido às mesmas pressões, ela acabou deixando de lado esse namorado e
se casando, aos quatorze anos, com um outro, portanto seu primeiro
marido, com quem teve quatro filhos.

Esse primeiro marido era mau, violento, mulherengo e infiel. O~casamento
não deu certo e Alice ficou sozinha com as crianças. Porém retomou o
relacionamento com seu primeiro namorado, que se tornou seu segundo
marido, quando ela já tinha vinte e sete anos.

O segundo marido e primeiro namorado frequentava uma igreja pentecostal
e era dependente de crack. Pouco tempo após o casamento, obrigou Alice a
fumar pedra. Quando ela recusou, ele encostou uma faca no seu pescoço e
ameaçou furá"-la caso ela não fumasse. Ela fumou sua primeira pedra e
disse que nada sentiu. Ele ficou furioso e gritou que ela não tinha
fumado direito. Obediente, ela fumou outra pedra, depois mais outra,
depois mais outra. Então sentiu.

Desde aquele momento Alice fumava crack exclusivamente com esse rapaz,
com quem viveu durante alguns anos. A~união conjugal, no entanto, não
deu certo devido ao comportamento violento dele, e também por não
aceitação da família dela.

Alice arrumou um terceiro marido que, sendo também crente e frequentador
de uma igreja evangélica, cometeu seu primeiro assalto e foi preso. O~casamento se desfez.

Alice continuou fumando crack e lembrava"-se sempre daquele que fora seu
primeiro namorado, e também lembrava"-se bem de sua iniciação no crack
com a ponta da faca encostada no pescoço.

Quando eu insisto em saber se Alice deseja afugentar a lembrança da faca
no pescoço, ela responde que não consegue e depois responde que não
quer.

Alice vive agora com seu quarto marido, que não bebe, não fuma, não usa
faca, não assalta e não é crente. Mas Alice não gosta dele. Considera"-o
``carrasco'' e ``ignorante'', e apenas o atura. Quando fuma sua habitual
pedra diária, Alice manda buscar o material na biqueira através de algum
oportuno portador, sem que o marido saiba.

Alice tem fumado sozinha e nunca esquece seu primeiro namorado, como se,
ao fumar, ela refizesse uma relação íntima com quem foi o ``homem de sua
vida'', num enlace que constelaria talvez a perfeita ``brisa'' assentada
na pedra!

Eu me pergunto se os sucessivos casamentos de Alice poderiam ser
considerados atos de traição! Imagino que sim. Principalmente no caso do
último casamento, pois esse último marido --- o ``carrasco'', o
``ignorante'' --- mal sabe que sua esposa faz surgir virtualmente um
rival que ela conhecera desde menino e lhe instilara uma louca paixão ao
lhe fazer ameaças, enquanto ela, passiva e resignada, cedia aos seus
avanços.

Sou levado a crer que Alice tenha na verdade apenas um consorte fixo,
uma vez que os outros foram amantes temporários. A~história infeliz de
Alice é uma troca curiosa de amantes em meio à invocação de um único
marido eleito --- presente ou ausente em corpo --- e eleito não apenas
para incentivar traições, mas para tornar possível uma descarada
fidelidade.

Alice tem um jeito algo esquizoide e depressivo. Se ela é uma mulher
simplória, é também uma mulher estranha e fechada. Com certeza herdou
uma predisposição genética para vários distúrbios, o que é exemplificado
nos parentes alcoolistas e depressivos, sem falar da filha que a imitou
em sua primeira tentativa de suicídio também aos treze anos.

É difícil concluir a respeito de qual seja mesmo o transtorno básico de
Alice. Mas tudo indica que, se Alice vem cedendo loucamente a pressões
que possuem um negativo dom alquímico de transformar violência doméstica
em fissura por crack, ela carrega a volúpia por uma brutalidade partindo
de algum macho a lhe instilar um vício na rabeira de um erotismo torto e
a lhe criar mais do que uma simples ``dependência química''. Porque não
se trata apenas de mais uma doentia paixão, porém de um desejo
ensandecido por um pobre diabo que cruzara seus caminhos desde uma
infância de namoricos inocentes até uma época em que ele --- com bizarro
charme e cínico deboche --- saía da igreja pentecostal rumo às vielas
dos ``noias'' de crack.

Eu me pergunto se esse rapaz seria uma vil imitação da obscura
contraparte de Eros, aquele com quem --- segundo o mito e de acordo com
um truque habilmente dissimulado pelo Oráculo de Delfos --- Psiquê
estaria aparentemente destinada a encontrar no alto da montanha?!

Ou talvez a história de Alice não seja nada disso, e um outro mito
explique tudo melhor. Mesmo porque, a história de Alice é um pouco
diferente da história de Psiquê, em que o monstro é simbólico e é um
hábil disfarce dos deuses para esconder uma outra face --- a beleza e o
esplendor de Eros.

Na mente de Alice o monstro é concreto e estupidamente real. Ao
contrário da bem aventurada Psiquê, Alice não viveu o alívio de
constatar uma condição ilusória como mera passagem e nem teve de
imaginar e temer o monstro para depois desfrutar da beleza de Eros. Eros
que, na trajetória infeliz de Alice, é cada vez mais distante, cada vez
mais invisível, improvável, ou mesmo impossível.

Alice, que nunca conheceu o verdadeiro amor, conheceu cedo o lado
falsamente inocente de um perverso rosto infantil. A~partir daí
manteve"-se fiel ao monstro a despeito de seus sucessivos maridos.

Ela manteve"-se loucamente fissurada tanto pelo monstro quanto pela
sombra dela própria, e também manteve"-se fissurada por um objeto mágico:
a pedra de crack chegando a ela incandescente --- em oferta sedutora ---
através dele, numa insistente e terrível conquista mediada pela ponta de
uma faca que produziu dor mas também produziu um estranho prazer.

\begin{center}\asterisc{}\end{center}
%\begingroup\small

\emph{Tenho visto mulheres estupidamente adictas em crack e em outras
drogas, em que a dependência surge através de um sórdido condicionamento
a algum parceiro amoroso ou a alguém próximo por vínculos familiares, e
de tal maneira que o uso da substância é um prolongado ritual de
preservação de uma relação suspeita.}

\emph{Há casos de ``luto'' em que a perda de um objeto amoroso é
substituída pela substância psicoativa. Como nestes exemplos seguintes:
o caso de uma moça que ``fumava'' seu ex-namorado que a tinha iniciado
no crack; o caso de uma mulher cujo marido dependente de crack lhe disse
que preferia o crack a ela, e a mulher, para testar uma diferença de
paixões, apaixonou"-se pelo crack e largou o marido; ou o caso de uma
mulher que detestava o álcool, perdeu o pai alcoolista, e depois da
morte do pai passou a embriagar"-se sozinha todos os dias fechada num
quarto sem ter prazer com a cachaça e pensando o tempo todo no pai.}

\emph{O~caso infeliz de Alice parece não fugir dessas adições viscosas
na rabeira de relacionamentos íntimos doentios. E~nesta história de
Alice não há dúvida a respeito da existência das chamadas comorbidades
psiquiátricas. Alice --- eu me lembro bem agora --- tinha um traço algo
esquizoide e tinha também uma reserva pessoal por conta de sua atitude e
postura arredias.}

\emph{Para mim ficou evidente, já no primeiro atendimento, uma síndrome
depressiva. A~história familiar dela trazia elementos indicando uma
vulnerabilidade genética à depressão e também à própria droga"-adicção,
como os dados da história do pai e da filha de Alice sugerem.}

\emph{Eu então especulei várias possibilidades, tive várias dúvidas e
fiquei tentado a pensar até num transtorno de personalidade. Mas não
transtorno de personalidade antissocial, é claro, e sim transtorno de
personalidade dependente.}

\emph{Por quê? Porque Alice me deu a impressão de ser aquela criatura
ausente de si própria, zerada em autoestima, uma pessoa portadora de
alguma ``síndrome de Eco'', necessitada de se vincular de maneira
doentia a um rol de criaturas, demonstrando ser uma parasita de almas
enquanto presa a uma dependência basal e intransitiva da qual a
dependência química é apenas consequência.}

\emph{Trata"-se, afinal de contas, de um caso complicado e que pode ser
visto de ângulos diversos. Que pode ser visto até conforme diversos
critérios científicos.}

\emph{A~maneira com que Alice se vinculava à pedra, ou seja, por meio da
nefasta intermediação de uma faca cutucada no pescoço e por meio do
grude ao rapaz desejado, é um exemplo bizarro de um condicionamento que
me lembra a reflexologia pavloviana.}

\emph{Não estou brincando, caro leitor. O~consorte de Alice e seu
primeiro namorado, aquele rapaz dividido entre a igreja e o crack, fez
de Alice uma imitação do cão de Pavlov. No lugar da campainha, a ponta
da faca; e no lugar da salivação, o gosto pela pedra, e pedra que
significava o rapaz transmutado na forma de objeto.}

\emph{Ou então, vejamos melhor: na associação de estímulos criou"-se um
prazer em meio à dor. A~adicção de Alice seguiu o curso de um ritual
masoquista por um mecanismo de reflexo condicionado, embora essa
rotulação de ``masoquismo'' seja apenas um chavão popular, imagino eu.}

\emph{O~que ocorre, na verdade, é uma possível estrutura defeituosa e
dependente na personalidade de Alice, cuja origem deve vir da infância.
Digo estrutura defeituosa que remonta a uma estranha paixão nos tempos
de criança e à fixação naquele que se tornaria seu consorte permanente
porém consorte ausente, a despeito das ``traições'' temporárias.}

\emph{Ironicamente, o caso de Alice se revelaria como uma poliandria
simbólica em que o sucessivo desfile dos maridos reforça uma fidelidade
àquele que é seu objeto compulsivo do desejo.}

\emph{Mas talvez ela nem soubesse que estaria sempre procurando o mesmo
consorte, e ademais não se trata de um caso de verdadeiro amor e sim de
uma volúpia sugerindo o que poderia ser um erotismo degradado.}

\emph{Puxa vida! Caro leitor, confesso que, naquela altura das minhas
especulações, eu já precisava recorrer aos meus conselheiros, e foi aí
que entrou em cena a mitologia grega --- tão providencial nesses casos
porque lá, como na obra de Shakespeare, estão todas as vicissitudes da
humanidade.}

\emph{De repente, o mito de Eros e Psiquê veio à tona, porém veio numa
inversão de polaridades, numa espécie de antimito de Eros e Psiquê.}

\emph{Percebi que a paixão de Alice era mesmo pelo monstro. Foi ao
monstro que uma Alice/Psiquê} fake \emph{se manteve fiel e fissurada.}

\emph{Raramente tenho visto exemplos tão dramáticos mostrando uma
descarada situação de dependência para além da chamada dependência
química; raramente tenho visto com tanta clareza a ambivalência da dor e
do prazer num convívio docemente infernal.}

\emph{Alice acabou sendo uma anti-heroína cuja saga não teria sido a
bem"-aventurança predita por oráculo amigo, e sim uma maldição predita
por oráculo inimigo.}

\emph{Mas quando, no primeiro atendimento, eu vi aquela Alice em sua
morbidez e contundente realidade minhas impressões produziram um certo
incômodo. E~pelo fato de que dela exalava um empobrecido e simplório
mistério, não foi fácil assimilar o caso para lhe descobrir ligações
míticas e perceber o cinismo das traições simbólicas e a paixão pelo
monstro.}

\emph{Mas creio que segui meu caminho de} ``dias"-gnosis'' \emph{pelo
duplo caminho da psicopatologia e da mitologia grega, até perceber a
possibilidade de um encontro desses caminhos na complexidade de Alice.}

\emph{Eu chego até a dizer que este conto de Alice parece uma tragédia
peculiar de enredo recorrente, dentro do qual, na trilha monótona da
repetição e da compulsão, Alice mantém frieza, desencanto e automatismo gestual --- tudo isso a me sugerir uma condição mais complexa do que
uma simples depressão, ou uma simples adicção.}

\emph{Por isso eu creio que o melhor recorte diagnóstico provisório para
o caso de Alice venha de uma agonia trágica, ou seja, de uma paixão
canhestra e doentia levando ao grude do monstruoso apego. No transcurso
dessa agonia a pedra de crack é a pedra no meio do caminho e é a pedra
que liga; a pedra de crack é o vínculo reflexológico que pode ser
decomposto em fases sequenciais; como em outros casos, de outros
adictos, a pedra é também mais um sofrimento da ilusão virando espúrio
prazer da busca eterna da ``brisa''.}

\emph{Alice, enfim, virou paradoxo de si mesma, virou quase uma Eco com
carência absoluta de um Narciso vulgar que não é definitivamente ``flor
que se cheire''.}

\emph{Uma Eco esquizoide versus um Narciso degradado sem beleza e a quem
--- ambos na verdade --- os deuses não teriam se apiedado, ou teriam até
deixado no esquecimento. ~}
%\endgroup