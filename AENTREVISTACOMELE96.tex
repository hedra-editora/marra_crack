\chapterspecial{A entrevista com ele}{}{}
 

A ``viagem sem fim'' continuava, capítulo por capitulo, depois que ele
ia dormir. Mas a viagem era interrompida quando ele acordava, e retomava
no outro dia de noite.

Os medos e pressentimentos só melhoravam quando ele dormia ou quando
surgia uma nova estória dentro da estória sem fim.~

Agora, porém, não há mais sonhos. Sobra apenas um vazio.~

Eu pergunto o que ele acha de toda a estória sem fim, e se ele acha que
aconteceu. Ele até acha que aconteceu, mas pensa também que foi sonho.
Daí fica na dúvida porque tudo parecia muito real.~

Ele não sabe dizer se a estória pode continuar. Ainda acredita que o
objetivo final da estória era ele casar"-se com a mulher loira, quando
então a ``estória sem fim'' deveria ter um fim.~

Ou ele acredita que a estória possa existir em outro tempo. Ou acredita
que a estória possa existir em outro lugar.

Na verdade, ele tem vontade de entrar de novo na estória porque ela traz
a paz e um certo prazer.~

No entanto, ele se lembra agora de um outro final.

Certo dia ele foi se aproximando da mãe logo depois que tinha acordado.

Ele ainda estava meio confuso.~

De repente ele sentiu que a mãe era um obstáculo em sua jornada rumo ao
seu casamento com a mulher loira.~

Tudo aconteceu numa época em que ele tinha unhas longas e cabelos
longos. Era um rapaz assustador, grande e forte.~

Ele segurou o pescoço da mãe.~

Sua irmã, policial militar, chegava naquele momento em casa de manhã.~

Sua irmã pensou que ele estivesse querendo matar a mãe por causa das
drogas.~

Sua irmã lhe deu um tiro que acertou seu pescoço.~

O tiro interrompeu o sonho porque depois do tiro os capítulos da
``estória sem fim'' foram encerrados.~

Ele ficou internado muito tempo. Quase morreu.~

Aquelas imagens do sonho nunca mais voltaram, e hoje ele não vê como
retomar a estória a não ser, talvez, sonhando o sonho.~

Mas ele ainda se sente um herói na lembrança do sonho, e eu pergunto se
a lembrança do sonho o assusta. Ele responde que não. O~que o assusta é
a realidade.~

Ele confessa que gostaria de retomar a ``viagem sem fim'', que adoraria
entrar de novo na estória, capítulo por capítulo, viajando a cada noite
depois de cada crise diurna de agonia, crise que era aliviada como se
cada capítulo fosse um longo e prolongado mesclado. ~~

\subsubsection{}
